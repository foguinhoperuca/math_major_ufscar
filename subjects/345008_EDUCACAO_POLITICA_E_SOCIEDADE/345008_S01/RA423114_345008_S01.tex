%% Copyright (C)  2012 Jefferson Campos
%% Permission is granted to copy, distribute and/or modify this document
%% under the terms of the GNU Free Documentation License, Version 1.3
%% or any later version published by the Free Software Foundation;
%% with no Invariant Sections, no Front-Cover Texts, and no Back-Cover Texts.
%% A copy of the license is included in the section entitled "GNU
%% Free Documentation License".

\documentclass[a4paper,12pt]{article}
\usepackage[brazilian]{babel}
\usepackage[utf8]{inputenc}
\usepackage[T1]{fontenc}
\usepackage{hyperref}
\usepackage{url}
\usepackage[pdftex]{graphicx}
\newcommand{\HRule}{\rule{\linewidth}{0.5mm}}

\begin{document}

\section {Capítulo 05 - Sociedade Para Sociologia Crítica.}

\begin{itemize}
\item Biografia de Karl Marx: nascimento, profissão, tirania Frederico IV;
\item Durkheim (sociologia leva sociedade capitalista à perfeição) vs Marx (sociedade imperfeita: tendência a luta de classes)
\item Durkhein (moral social) vs Marx (consciência das pessoas e a formas de organização do trabalho);
\item Marx: 
\begin{itemize}
\item Organização da produção como grande depósito de mercadorias;
\item Além da relação de troca, está a relação de produção
\item Duas classes sociais básicas: proletários vs Capitalistas
\item Mais-valia: excedente de produção do proletário que o capital se apropria para reproduzir-se;
\item Somente o trabalho produz o lucro. Os demais bens apenas transformam-se;
\item Contradição básica do capitalistmo: exploração do proletário pelo capital (Proletário: pobreza. Capital: riqueza);
\item Por excelência, a sociedade capitalista é uma sociedade de violência;
\item Proposta: ao invé de aperfeiçoar o capitalismo, atacar as raizes do problema (contradição): levar a sociedade rumo ao socialismo;
\item Para materializar a proposta, é necessário a organização dos proletários em luta de seus direitos;
\end{itemize}
\item Durkheim: ciência sociológica + educação: reforma do capitalismo. Marx: organização e luta do proletário leva à uma sociedade melhor;
\end{itemize}


\section{Capítulo 06 - Educação para Sociedade Crítica.}

\subsection{Ideologia e sua Relação com a Educação.}

\begin{itemize}
\item Marx: reconhece a importância da educação mas não trata diretamente o tema;
\item Elemento fundador da sociedade: o trabalho (transformação da natureza);
\item Relações sociais: através do trabalho, as pessoas passam a pensar e desenvolvem a consciência e comunicação de idéias - interpretação da sociedade e atividades práticas;
\item Marx: não é a consciência que explica a sociedade mas determinada maneira de transformar a natureza gerando uma consciência.
\item Contradição: nem sempre a realidade social corresponde àquilo que pensamos sobre a realidade. Por quês?
\item Ocultação da realidade social (exploração): porque certos valores e idéias surgem para encobrir essa realidade?
\item Difereça da visão da mesma relação entre as classes.
\begin{itemize}
\item Capitalista: reforça os aspectos positivos do trabalho (fonte de lucro);
\item Proletário: reforça os aspectos negativos (fonte de pobreza, privações, salários baixos, etc).
\end{itemize}
\item A visão capitalista predomina e aparece como única realidade - facilita a dominação do proletário;
\end{itemize}

\subsection{Ideologia e sua Relação com a Educação.}

\begin{itemize}
\item Durkheim: escola reprodutora da moral social;
\item Marx: escola é instrumento de reprodução dos interesses da classe empresárial;
\item A escola, em uma sociedade como tal, reproduz a divisão e o conflito;
\item Durkheim: educação una. Marx: educação é por classe;
\item Marx: a escolaridade para a classe trabalhadora tem 2 objetivos: transmissão da ideologia e preparação do individuo para o trablho;
\item Marx: a escolaridade para a classe empresarial é superior (qualidade)
\item O conhecimento é fonte de poder. Logo, faz sentido a classe capitalista tentar dominar ele;
\end{itemize}

\subsection{Conclusão.}
\begin{itemize}
\item Marx: escola como instituição sob o controle da classe empresarial;
\item Proletário não deve negar a educação (pelo contrário);
\end{itemize}

\section{Licença.}

Copyright (C) 2012 Jefferson Campos.\\
Permission is granted to copy, distribute and/or modify this document\\
under the terms of the GNU Free Documentation License, Version 1.3\\
or any later version published by the Free Software Foundation;\\
with no Invariant Sections, no Front-Cover Texts, and no Back-Cover Texts.\\
A copy of the license is included in the section entitled "GNU\\
Free Documentation License".

\end{document}

