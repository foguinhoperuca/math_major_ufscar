%% Copyright (C)  2012 Jefferson Campos
%% Permission is granted to copy, distribute and/or modify this document
%% under the terms of the GNU Free Documentation License, Version 1.3
%% or any later version published by the Free Software Foundation;
%% with no Invariant Sections, no Front-Cover Texts, and no Back-Cover Texts.
%% A copy of the license is included in the section entitled "GNU
%% Free Documentation License".

\documentclass[a4paper,12pt]{article}
\usepackage[brazilian]{babel}
\usepackage[utf8]{inputenc}
\usepackage[T1]{fontenc}
\usepackage{hyperref}
\usepackage{url}
\usepackage[pdftex]{graphicx}
\newcommand{\HRule}{\rule{\linewidth}{0.5mm}}

\begin{document}

\begin{titlepage}

\begin{center}

\includegraphics[width=0.15\textwidth]{./ufscar.jpg}\\[1cm]    

\textsc{\LARGE UNIVERSIDADE FEDERAL DE SÃO CARLOS}\\[1.5cm]

\textsc{\Large UFSCAR}\\[0.5cm]

\HRule \\[0.4cm]
{ \huge \bfseries Prova de continuídade para $(g \circ f)(x,y)$.}\\[0.4cm]

\HRule \\[1.5cm]


\begin{minipage}{0.4\textwidth}
\begin{flushleft} \large
\emph{Autor:}\\
Jefferson \textsc{Campos}\\
RA 423114\\
\end{flushleft}
\end{minipage}
\begin{minipage}{0.4\textwidth}
\begin{flushright} \large
\emph{Supervisor:} \\
Dra. Silvia Maria Simões de  \textsc{Carvalho}
\end{flushright}
\end{minipage}

\vfill

{\large \today}

\end{center}

\end{titlepage}


\section {Proposta 01: Utilizando o Winplot para Resolução de Exercício de Função do $2\,^{\circ}$ Grau.}

\subsection{Conteúdo.}

Função do $2\,^{\circ}$ Grau e definição dos vértices da parábola.

\subsection{Etapa Escolar.}

$2\,^{\circ}$ bimestre da 1\textordfeminine série do Ensino Médio.

\subsection{Enunciado.}

Esboce o gráfico da função $y = 2x^2$ -$ 3x + 1$ e determine:

\newcounter{winplot_enunciado}
\begin{list}{\alph{winplot_enunciado}) }{\usecounter{winplot_enunciado}}
\item Determine as raízes da função.
\item Determine as coordenadas do vértice.
\item Interprete o significado de ${y_v}$ [valor máximo | valor mínimo] da função.
\item Determinar a intersecção da função com o Eixo y.
\end{list}

\subsection{Roteiro de Resolução.}

A proposta para esta sequência didática tem como ponto de partida a familiaridade com o software Winplot.

A resolução do \textbf{item a} leva em conta a familiaridade com a fórmula de Bháskara, a utilização das linhas de grade para uma melhor visualização e o uso da funcionalidade inventário para “imprimir” na tela a equação.

A resolução do \textbf{item b} leva em consideração o conhecimento da fórmula dos vértices. A ferramenta é capaz de gerar uma aproximação visual que vai ajudar o aluno em um primeiro momento confirmar/refutar a solução calculada no papel. Sendo incapaz, no software, de visualizar os pontos encontrado nos cálculos manuais inseridos na reta do gráfico, o aluno tem um feedback instântaneo que os seus cálculos não estão adequados. Caso seja capaz de identificar os pontos na reta, apesar de não ser uma garantia de corretude para o cálculo, irá incentivar o aluno a continuar a desenvolver a solução do problema.

No \textbf{item c}, o aluno tem uma clara ajuda do gráfico que é desenhado pelo software. Com ele é simplificada a tarefa de determinar se o valor de ${y_v}$ é o mínimo ou máximo da função.

No \textbf{item d}, a intersecção da curva com o eixo y, depende da resolução do item a. assim o item d atua como uma forma de visualizar o resultado, em termos de gráfico, do \textbf{item a}, reforçando o aprendizado.

\subsection{O Papel do Software em um Ambiente de Aprendizagem Construtivista.}

O uso de software no processo de aprendizagem é uma ferramenta importante que faz com que o aluno tenha a oportunidade de agir e refletir sobre o problema dado.

O uso do software também implica em uma postura diferente, não só do professor mas também do aluno. É esperado que ambos tenham, ao menos, uma familiaridade com o uso de um computador do tipo PC.

Esta atividade, também, requer que o aluno tenha um conhecimento prévio sobre equação do 2 grau e na sua fórmula para resolução.

O uso da linha de grade facilita a visualização dos pontos do gráfico e dá uma noção de visual de quanto a reta “se extende”. Isso simplifica o processo de determinar o valor de um ponto por inspeção visual, sendo muitas vezes suficiente para resolução de determinados problemas. Esse uso da grade estimula a sedimentação do conceito de lugar geométrico.

Finalmente, este exercício visa ajudar o aluno a obter uma solução gráfica para o problema de determinação dos vértices de uma parábola.

Vale salientear que o uso do software deve ser apoiado pela resolução das euqações e demais cálculos via “lápis e papel”. O uso do lápis e papel de forma complementar irá contemplar o desenvolvimento de conceitos de escala.

\subsection{Competências e Habilidades.}

O aluno deve ser capaz de:
\newcounter{winplot_competencia}
\begin{list}{\alph{winplot_competencia}) }{\usecounter{winplot_competencia}}
\item Identificar uma função do 2 grau;
\item Reconhecer no software cada eixo do plano cartesiano, bem como as coordenadas de um ponto;
\item Verificar se um ponto satisfaz ou não a equação da reta;
\item Observar a forma gráfica de uma função do 2 grau;
\item Definir as propriedades de coordenadas dos vértices e determinar a sua solução;
\item Resolver uma equação do segundo grau e determina as raízes da função;
\item Aplicar adequadamente a escala na representação do plano cartesiano;
\item Representar pontos no plano cartesiano;
\end{list}

\section {Proposta 02: Utilizando o Geogebra para o Estudo Dos triângulos Escalene, Equilátero e Isósceles.}

\subsection{Conteúdo.}

Relação entre ângulos de um triângulo.

\subsection{Etapa Escolar.}

$2\,^{\circ}$ bimestre da 6{\textordfeminine} série do Ensino Fundamental II.

\subsection{Enunciado.}

Desenhe, no geogebra, os triângulo e determine:

\newcounter{geogebra_enunciado}
\begin{list}{\alph{geogebra_enunciado}) }{\usecounter{geogebra_enunciado}}
\item Desenhe um triângulo isósceles, escaleno e equilátero.
\item Altere as coordenadas do triângulo (escaleno por exemplo), uma por vez. O que acontece com o valor dos ângulos?
\item Tomando o valor de dois ângulos, aleatoriamente, de um triângulo qual deve ser o valor do terceiro ângulo? Porquê?
\end{list}

\subsection{Roteiro de Resolução.}


A proposta para esta sequência didática tem como ponto de partida a familiaridade com o software geogebra.

A resolução do \textbf{item a} é necessário desenhar os três triângulos pedidos na ferramenta. É necessário atentar que para todos os triângulos desenhados, com a ferramenta de ângulo defina os 3 ângulos internos do triângulo. Depois, ir na opção propriedades e marcar a opção para exibir “nome e valor” e dar o nome apropriado à cada triângulo [escaleno|isósceles|equilátero]. Para tanto é necessário:

\subsubsection{Triângulo Escaleno.}

\newcounter{geogebra_escaleno}
\begin{list}{\arabic{geogebra_escaleno}) }{\usecounter{geogebra_escaleno}}
\item Com a ferramenta polígno desenhe um triângulo (é necessário criar três pontos e finalmente clicar no ponto inicial para que o triângulo seja “fechado”);
\end{list}

\subsubsection{Triângulo Equilátero.}

\newcounter{geogebra_equilatero}
\begin{list}{\arabic{geogebra_equilatero}) }{\usecounter{geogebra_equilatero}}
\item Pela opção segmento definido por dois pontos crie um segmento DE;
\item A partir desse segmento utilize a opção círculo definido pelo centro e crie um circunferência com centro no ponto D e raio com o comprimento do ponto D até o ponto E. Repita o procedimento, partindo do ponto E.
\item Com a opção interseção de dois objetos crie um ponto F na interseção entre os dois círculos.
\end{list}

\subsubsection{Triângulo Isóceles.}

\newcounter{geogebra_isoceles}
\begin{list}{\arabic{geogebra_isoceles}) }{\usecounter{geogebra_isoceles}}
\item Utilize a opção Segmento definido por dois pontos para criar um segmento GH;
\item Utilize a opção de ponto médio ou centro para para criar um ponto I;
\item A opção reta perpendicular deve ser utilizada para criar uma reta perpendicular ao segmento GH;
\item A opção novo ponto deve ser utilizada para criar um novo ponto J na reta h;
\item Utilize a opção polígno para desenhar os triângulo;
\end{list}


A resolução do \textbf{item b} será dada com a alteração da coordenada de uma vértice do triângulo. Espera-se que o aluno note que a alteração implique em mudança no valor o ângulo adjacente e que essa mudança é refletida nos demais ângulos do triângulo.

No \textbf{item c}, o aluno deve verificar qua a soma dos ângulos de um triângulo é 180º independente do valor de cada ângulo.

\subsection{O Papel do Software em um Ambiente de Aprendizagem Construtivista.}

O uso de software no processo de aprendizagem é uma ferramenta importante que faz com que o aluno tenha a oportunidade de agir e refletir sobre o problema dado.

O uso do software também implica em uma postura diferente, não só do professor mas também do aluno. É esperado que ambos tenham, ao menos, uma familiaridade com o uso de um computador do tipo PC.

O uso da ferramenta geogebra facilita o processo de descoberta e interatividade, fazendo que o aluno consiga agir sobre o objeto matemático (no caso o triângulo) e imediatamente perceber as consequências.

\subsection{Competências e Habilidades.}

O aluno deve ser capaz de:
\newcounter{geogebra_competencia}
\begin{list}{\alph{geogebra_competencia}) }{\usecounter{geogebra_competencia}}
\item Identificar três tipos de triângulos: isósceles, escaleno e equilátero;
\item Perceber as implicações na alteração dos valores de um vértice ou de um ângulo;
\item Reconhecer que a soma dos ângulos internos é $180\,^{\circ}$;
\end{list}

\section{Licença.}

Copyright (C)  2011 Jefferson Campos.\\
Permission is granted to copy, distribute and/or modify this document\\
under the terms of the GNU Free Documentation License, Version 1.3\\
or any later version published by the Free Software Foundation;\\
with no Invariant Sections, no Front-Cover Texts, and no Back-Cover Texts.\\
A copy of the license is included in the section entitled "GNU\\
Free Documentation License".

\nocite{website:aula_geogebra}
\nocite{texbook:matematica_em}
\bibliographystyle{ieeetr}
\bibliography{P2_423114}

\end{document}