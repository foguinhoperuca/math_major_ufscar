%% Copyright (C)  2012 Jefferson Campos
%% Permission is granted to copy, distribute and/or modify this document
%% under the terms of the GNU Free Documentation License, Version 1.3
%% or any later version published by the Free Software Foundation;
%% with no Invariant Sections, no Front-Cover Texts, and no Back-Cover Texts.
%% A copy of the license is included in the section entitled "GNU
%% Free Documentation License".



\documentclass[a4paper,12pt]{article}
%%\usepackage{amsmath}
\usepackage[brazilian]{babel}
\usepackage[utf8]{inputenc}
\usepackage[T1]{fontenc}
\usepackage{hyperref}
\usepackage{url}
\usepackage[pdftex]{graphicx}
%%\usepackage{mathtools}
\usepackage{amssymb}
\usepackage[usenames]{color}
\usepackage[version=3]{mhchem}
\title{ Prova de continuídade para $(g \circ f)(x,y)$.}
\author{Jefferson Campos - RA 423114}
\date{May, 02 2012}
\newcommand{\HRule}{\rule{\linewidth}{0.5mm}}

\begin{document}

\begin{titlepage}

\begin{center}

\includegraphics[width=0.15\textwidth]{./ufscar.jpg}\\[1cm]    

\textsc{\LARGE UNIVERSIDADE FEDERAL DE SÃO CARLOS}\\[1.5cm]

\textsc{\Large UFSCAR}\\[0.5cm]

\HRule \\[0.4cm]
{ \huge \bfseries Prova de continuídade para $(g \circ f)(x,y)$.}\\[0.4cm]

\HRule \\[1.5cm]


\begin{minipage}{0.4\textwidth}
\begin{flushleft} \large
\emph{Autor:}\\
Jefferson \textsc{Campos}\\
RA 423114\\
\end{flushleft}
\end{minipage}
\begin{minipage}{0.4\textwidth}
\begin{flushright} \large
\emph{Supervisor:} \\
Dra. Silvia Maria Simões de  \textsc{Carvalho}
\end{flushright}
\end{minipage}

\vfill

{\large \today}

\end{center}

\end{titlepage}


\section{Prova de continuídade para $(g \circ f)(x,y)$.}

\subsection{Teorema.}
$f(x,y)$ é contínua em $(x_0,y_0)$\\
e\\
$g(x,y)$ é contínua em $f(x_0,y_0)$\\
então\\
$h(x,y) = g(f(x))$ é contínua em $(x_0,y_0)$\\

%%$\lim_{(x,y)\to (x_0,y_0)}{\frac{e^x-1}{2x}}$

\subsection{Prova.}

Como $f(x,y)$ é contínua em $(x_0,y_0)$ pela definição do teorema podemos escrever:\\
$f(x,y) = f(x_0,y_0)$\\
Temos que:\\
$\lim_{(x,y)\to (x_0,y_0)}{f(x,y)} = f(x_0,y_0)$
Uma vez que $g$ é contínua em $b = f(x_o,y_0)$ podemos aplicar o ``teorema 8'' para obter:\\
$\lim_{(x,y)\to (x_0,y_0)}{g(f(x,y))} = g(f(x_0,y_0))$\\
que é justamente a definição do teorema.

\subsection{Teorema 8.}

O teorema é definido, segundo James Stweart (Cálculo Volume 1):\\
%%Seja $f$ contínua em b e $\lim_{(x,y)\to (x_0,y_0)}{g(f(x,y))} = g(f(x_0,y_0))$ então o $\lim_{(x,y)\to (x_0,y_0)}{g(f(x,y))} = g(f(x_0,y_0)) = f(b)$

$\lim_{x \to a}{f(g(x))} = f(\lim_{x\to a}{g(x)})$\\

\section{Licença.}

Copyright (C) 2012 Jefferson Campos.\\
Permission is granted to copy, distribute and/or modify this document\\
under the terms of the GNU Free Documentation License, Version 1.3\\
or any later version published by the Free Software Foundation;\\
with no Invariant Sections, no Front-Cover Texts, and no Back-Cover Texts.\\
A copy of the license is included in the section entitled "GNU\\
Free Documentation License".

\nocite{texbook:STEWART}
\bibliographystyle{ieeetr}
\bibliography{prova_continuidade}

\end{document}

