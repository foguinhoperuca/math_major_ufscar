\documentclass[a4paper,12pt]{article}
\usepackage[brazil]{babel}
\usepackage[utf8]{inputenc}
\usepackage[usenames]{color}
\usepackage[version=3]{mhchem}
\title{Resenha - Livro o Monge e o Executivo.}
\author{Jefferson Campos - RA 423114}
\date{Novembro, 30 2012}

\begin{document}

\maketitle
\newpage

\section{Sobre o Livro.}

O Monge e o Executivo é um livro trata basicamente de liderança e é inspirado na vida e ensinamento de Jesus Cristo. O livro passa por temas como poder, hierarquia, relacionamentos, quebra de paradigmas entre outros tendo como fundo um alto executivo que passa uma temporada em um retiro espiritual (monastério) onde ele, juntamente com Simeão (anteriomente um alto executivo e agora um líder espiritual) repassam esses temas através de conversas e aulas.

\section{Minha Opnião.}

O livro trata os temas de forma genérica, utilizando-se do senso comum onde é difícil ter uma posição contrária como, por exemplo, em relação ao fato que um líder deve ouvir as pessoas com atenção e tratá-las com atenção e respeito. O autor também faz uma ``ponte'' entre teorias da psicologia como hierarquia das necessidades humanas de Abraham Maslow e alguns outros temas para reforçar o seu discurso e torná-lo algo mais rigoroso, tal qual é prática em um estudo científico, por exemplo. Entretanto, o livro continua à pecar por apoiar-se no senso comum. No fim, o livro usa um discurso com carater religioso para justificar práticas, como a servidão, que são de interesses de uma elite dominante e capitalista.

\section{Relação com Gestão Escolar Democrática.}

Entretanto, apesar dessa natureza do discurso do livro, pode se (guardada as devidas proporções e principalmente o julgamento particular de cada um) fazer um paralelo com a gestão escolar, especialmente a gestão escolar democrática. Particularmente, podemos fazer um paralelo com o livro os trecho que tratam sobre os relacionamentos entre as pessoas e principalmente a capacidade de um diretor de uma escola ouvir de forma sincera os outros membros da comunidade e levar em consideração as suas necessidades em suas decisões. Outro ponto que merece ser mencionado é a relação entre o comportamento e determinado problema. Na visão do autor, o comportamente é apenas uma manifestação à um problema que está ocorrendo, independente de que a comunidade esteja ciente ou não.

\section{Conclusão.}

Conforme mencionado anteriormente, o livro trata de temas de forma superfícial e baseado no senso comum sendo difícil posicionar-se contra. Esta é basicamente a receita de um livro de auto-ajuda (que este livro não deixa de ser). Entretanto, se relevarmos esses pontos, podemos extrair algumas lições e fazer um paralelo à gestão escolar democrática, conforme exposto anteriormente.


\end{document}
