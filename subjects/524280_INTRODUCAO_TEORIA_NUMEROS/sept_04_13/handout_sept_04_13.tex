%% Copyright (C)  2013 Jefferson Campos.
%% Permission is granted to copy, distribute and/or modify this document
%% under the terms of the GNU Free Documentation License, Version 1.3
%% or any later version published by the Free Software Foundation;
%% with no Invariant Sections, no Front-Cover Texts, and no Back-Cover Texts.
%% A copy of the license is included in the section entitled "GNU
%% Free Documentation License".

\documentclass[a4paper,12pt]{article}
\usepackage[brazilian]{babel}
\usepackage[utf8]{inputenc}
\usepackage[T1]{fontenc}
\usepackage{hyperref}
\usepackage{url}
\usepackage{amssymb}
\usepackage{amsthm}
\usepackage[pdftex]{graphicx}
\newcommand{\HRule}{\rule{\linewidth}{0.5mm}}

\newtheorem{add}{Adição}
\newtheorem{mult}{Multiplicação}
\newtheorem{dist}{Distributiva}
\newtheorem{prop}{Proposição}
\newtheorem{definit}{Definição}
\newtheorem{ax}{Axioma}

\begin{document}

\section{Definições Básicas.}

Assume-se que os elementos do conjunto dos inteiro $Z$ satisfazem os seguintes axiomas, $\forall a, b, c \in Z$:

\begin{add} % D3
  Seja $a \in Z$. O valor absoluto (ou módulo) de $a$, $|aA$, é definido por:\\
  $$
  |x|=\left\{
    \begin{array}{rc}
      a,&\mbox{se}\quad a\ge 0,\\
      -a, &\mbox{se}\quad a<0.
    \end{array}\right.
  $$

obs:

1 - se $a \leq b$ e $b \leq c$, pode-se escrever $a \leq b \leq c$\\
2 - $\forall a \in Z, |a| \ge 0$\\

EX:

$|3| = 3$ pois $3 \geq 0$\\
$|-2| = -(-2) = 2$ pois $-2 \leq 0$
\end{add}

\begin{add} % Proposição 4
  $\forall a, b,c \in Z$\\

  1 - $|a| \ge 0 $ e $|a| = 0 \iff a = 0$\\
  2 - $-|a| \leq a |leq |a|$\\
  3 - $|-a| = |a|$\\
  4 - $|a \cdot b| = |a|\cdot|b|$\\
  5 - $|a| \leq b  \iff -b \leq a \leq b$\\
  6 - $|a + b| \leq |a| + |b|$ - desigualdade triangular\\
  7 - $ ||a| - |b|| \leq |a| - |b|$\\

  Sejam | parcialmente ordenados mediante a relação $\leq$ e $A $ está contido em $Z$ com $A \neq \emptyset$
\end{add}

\begin{add} % Proposição 5
  1 - $L \in Z$ é  um limite superior de $A  \iff (\forall x \in A \Longrightarrow x \leq L) $\\
  % 2 - $$\\ % copiar manualmente

Observação:

1 - se $L \in Z$ é limite superior de A diz-se que A é limitado superior\\
2 - Se $l \in Z$ é limite inferior de A diz-se que A é limitado inferior\\
\end{add}

\begin{add} % Definição
  1 - $M \in A$ é um máximo de $A  \iff (\forall x \in A \Longrightarrow x \leq M) $\\
  2 - $m \in A$ é um mpinimo de $A \iff (\forall x \in A \Longrightarrow x \geq M) $ \\

  Observação:
  
  $M = max(A) \in A$
  $m = mix(A) \in A$
\end{add}

\begin{add} % Proposição 5 - de novo?
  1 - Se $M = max(A)$ então $M$ é único.\\
  2 - $m  = min(A)$ então $,$ é único \\
\end{add}

O axioma abaixo pe ocnhecido como ``Princípio da Boa Ordem (POB)''. Todo conjunto não-vazio de inteiros não-negativos contém um mínimo.Em símbolos temos:\\
$A \subset Z$ e $A \neq \emptyset$\\
Se $\forall x \in A, x \ge 0$ então existe $m = min(A)$

\begin{add} % Proposição 6
  Seja $a \in Z$\\
  Se $0 \leq a \leq 1$ então\\
  $a = 0$ ou $a = 1$
\end{add}
\begin{proof}
  Estratégia de Demonstração: Reduction ad Absurdum\\

  1 - $v(p) =V$
  2 - Suponha por absurdo que $v(\neg q) = V (\neg q \iff \neg (q_1 \vee q_2) \iff \neg q_1 \wedge \neg q_2)$\\
  3 - Desenvolva a teoria a partir de 2\\
  4 - Obtenha uma proposição $r$ tal que $r \iff \neg p$ (absurdo!)
\end{proof}

\begin{add} % Proposição 7
  Propriedade de Arquimediana:\\
  Seja $A = \{x \in Z | x \geq 0\}$\\
  Se $a, b \in A$ então\\
  $(\exists n \in A)(n \cdot a > b)$
\end{add}
\begin{proof}
  A estratégia utilizada é a redução ao absurdo.\\
  \begin{equation}
    Por hipótese é válido que a, b \in A (1)\\
    Suponha por absurdo que seja válido:\\
  \end{equation}
  \begin{equation}
    \neg q: (\forall n \in A)(n \cdot a \leq b) (2)\\
    Daí constrói-se o seguinte conjunto, considerando (2):\\
    n \cdot a \leq b \Longrightarrow 0 \leq b - n \cdot a\\
    assim temos:\\
  \end{equation}
  \begin{equation}
    S = \{b - n \cdot a | n \in A\} (3)\\
    Assim \forall x \in S tem-se x = b -n \cdot a \leq 0, n \in A. Daí pelo P.B.O:\\
  \end{equation}
  \begin{equation}
    \exists m = min(S) sendo m \in S (4)\\
    logo, \\
  \end{equation}
  \begin{equation}
    m = b = r \cdot a (5)\\
    para algum r \in A. considerando m' \in S | m' = b - (r + 1 ) \cdot a para algum r \in A. Assim temos:\\
  \end{equation}
  \begin{equation}
    m' = b - (r \cdot a + a)\\
    = b + (-r \cdot a - a)\\
    = (b - r \cdot a) - a = m - a\\
    \Longrightarrow m' = m - a
  \end{equation}

  Por (1) $a \in A$, assim $a > 0$\\
  $m - a < m$\\
  Daí\\
  $m' < m$ (6)\\

  Por (4) $m - min(S)$, logo pela definição 5 item 2\\
  $m \in S$ e $m\leq x \forall x in S$ (7)\\
  Por (6) e (7) e usando a transitividade de (O3) tem-se que:\\
  $m' < x, \forall x \in S$.\\
  Daí pela definição 5 item 2 temos que:\\
  $m' = min(S)$ (8)\\
  Tem se:\\
  $
  m = min(S)\\
  m' = min (S)
  $

  4th - Mas isso é um absurdo, pois o mínimo de S é único, pela proposição (5). Logo o absurdo foi considerar $n \cdot a < b, \forall a \in A$.\\
  Portanto, $(\exists n \in A)(n \cdot a > b)$
\end{proof}

% Proposição 7 Possui: PBO; definição 5 item 2; O3; Proposição 5; A base da prova é a prova por contradição.

\begin{add} % Proposição 8
  Seja $A \subset Z \wedge A \neq \emptyset$\\
  Se $A$ é limitado inferiormente então $(\exists m \in A)(\forall x \in A \longrightarrow x \geq m)$
\end{add}
\begin{proof}
  Dica: $S = \{x - l | \forall x \in A\}$
\end{proof}




\nocite{*}
\bibliographystyle{ieeetr}
\bibliography{handout_sept_04_13}

\end{document}