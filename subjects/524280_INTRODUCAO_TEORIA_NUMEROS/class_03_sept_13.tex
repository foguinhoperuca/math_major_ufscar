%% Copyright (C)  2013 Jefferson Campos.
%% Permission is granted to copy, distribute and/or modify this document
%% under the terms of the GNU Free Documentation License, Version 1.3
%% or any later version published by the Free Software Foundation;
%% with no Invariant Sections, no Front-Cover Texts, and no Back-Cover Texts.
%% A copy of the license is included in the section entitled "GNU
%% Free Documentation License".

\documentclass[a4paper,12pt]{article}
\usepackage[brazilian]{babel}
\usepackage[utf8]{inputenc}
\usepackage[T1]{fontenc}
\usepackage{hyperref}
\usepackage{url}
\usepackage{amssymb}
\usepackage{amsthm}
\usepackage[pdftex]{graphicx}
\newcommand{\HRule}{\rule{\linewidth}{0.5mm}}

\newtheorem{add}{Adição}
\newtheorem{mult}{Multiplicação}
\newtheorem{dist}{Distributiva}
\newtheorem{prop}{Proposição}
\newtheorem{definit}{Definição}
\newtheorem{ax}{Axioma}

\begin{document}

\section{Definições Básicas.}

Assume-se que os elementos do conjunto dos inteiro $Z$ satisfazem os seguintes axiomas, $\forall a, b, c \in Z$:

\begin{add} % A1
  Propriedade associativa para soma $+$:\\
  $a + (b + c) = (a + b) + c$
\end{add}
\begin{proof}
  é axioma...
\end{proof}

\begin{add} % A2
  Existência do elemento neutro aditivo:\\
  $\ni |0 \in Z| a + 0 = a = 0 + a$ % o \ni é o existe
\end{add}
\begin{proof}
  é axioma...
\end{proof}

\begin{add} % A3
  Existência do oposto:\\
  $\ni |-a \in Z| a + (-a) = 0 = (-a) + a$ % o \ni é o existe
\end{add}

\begin{add} % A4
  Propriedade Comutativa:\\
  $a + b = b + a$ % o \ni é o existe
\end{add}
\begin{proof}
  é axioma...
\end{proof}

\begin{mult} % M1
  propriedade associativa para ``$\cdot$'':\\
  $a \cdot (b \cdot c) =  (a \cdot b) \cdot c$
\end{mult}

\begin{mult} % M2
  Existência do elemento neutro multiplicativo:\\
  $\ni |1 \in Z| 1 \cdot a = a = a \cdot 1$ % o \ni é o existe
\end{mult}

\begin{mult} % M3
  Propriedade cancelativa:\\
  Sejam $\forall a, b, c \in Z, a a \neq 0$\\
  Se $a \cdot b = a \cdot c$ então $b = c$
\end{mult}

\begin{mult} % M4
  Propriedade comutativa:\\
  $a \cdot b = b \cdot a$
\end{mult}

\begin{dist} % D1
  Propriedade distributiva da multiplicação com relação à adição:\\
  $a \cdot (b + c) = a \cdot b + a \cdot c$\\
  $(b + c) \cdot a = b \cdot a + c \cdot a$
\end{dist}

\begin{dist} % D2
  Propriedade distributiva da multiplicação com relação à adição:\\
  $a \cdot (b + c) = a \cdot b + a \cdot c$\\
  $(b + c) \cdot a = b \cdot a + c \cdot a$
\end{dist}

\begin{prop} % (1)
  Propriedade do cancelamento para $+$:\\
  $(\forall a, b, c \in Z)(a + b = a + c \Longrightarrow b = c)$\\
\end{prop}
\begin{proof}
  $\forall a, b, c \in Z$ faz-se:\\ % +(-a)
  $a + b = a + c$\\ % A1
  $(-a) + (a + b) = (-a) + (a + c)$\\ % A3
  $((-a) + a) + b = ((-a) + a) + c$\\ % A2
  $0 + b = 0 + c$\\
  $b = c$
\end{proof}

\begin{prop} % (2)
  $(\forall a \in Z)(a \cdot 0 = 0 = 0 \cdot a)$
\end{prop}
\begin{proof}
  \begin{equation}
    a \cdot 0 = 0
  \end{equation}
  % por em uma tabela
  $\forall a, b, c \in Z a \cdot (b + c) = a \cdot b + a \cdot c$ por D1\\
  tomando $b = c = 0$ temos $a (0 + 0) = a \cdot 0 + a \cdot 0$ por (1)\\
  $a \cdot 0 $
   % esta incompleto

  analogamento faz-se para
  \begin{equation}
    \\
    0 = 0 \cdot a
  \end{equation}
\end{proof}

\begin{prop} %(3)
  $(\forall a, b, \in Z)(a \cdot b = 0 \Longrightarrow a = 0$ ou $b = 0)$
\end{prop}

\begin{prop} %(4)
  Regra dos sinais:\\
  $(\forall a, b, \in Z)$ tem-se:\\
  \begin{equation} % FIXME need reset the number
    -(-a) = a
  \end{equation}
  \begin{equation} % FIXME need reset the number
    -(-a) \cdot b = -(a \cdot b) = a \cdot (-b)
  \end{equation}
  \begin{equation} % FIXME need reset the number
    (-a) \cdot (-b) = a \cdot b
  \end{equation}
\end{prop}

\section{Relação de Ordem.}

\begin{definit}
 Chama-se relação binária de $E$ em $F$  todo subconjunto $R$ de $E \times F$, isto é, sejam $E, F$ conjuntos e $E \times F = \{(m, n) | m \in E e n \in F\}$ assim, se $R$ é uma relação binária de $E$ em $F$, tem-se que $R$ está contido  $ E \times F$ % FIXME colocar simbolo do "C" está contido
\end{definit}

\begin{definit}
 precisa copiar...
\end{definit}

Assim a relação ``menor ou igual'', $\leq$, é uma relação de ordem parcial sobre $Z$, sendo válido os axiomas:

\begin{ax}
  $(\forall  a \in Z)(a \leq a)$ reflexiva
\end{ax}
\begin{ax}
  $(\forall a, b \in Z)(a \leq b e b \leq a \Longrightarrow a = b)$ - anti-simétrica
\end{ax}
\begin{ax}
  $(\forall a, b, c \in Z)(a \leq b e b \leq c \Longrightarrow a \leq c)$ - transitiva
\end{ax}

Obs:

1 - $Z$ é parcialmente ordenado mediante a relação $\leq$
2 - Se $a \leq b e a \neq b$ neste caso utiliza-se $a \leq b$
3 - Se $a \leq b$ logo $a - b \leq 0$
4 - Usam-se o símbolo $b \geq a ou b > a$

% ---

São válidos válidos, também os axiomas:

\begin{ax}
  Dados $a, b, \in Z$ tem-se que $a < b ou b < a ou a = b$ - tricotomia
\end{ax}

\begin{ax}
  $(\forall a, b, c \in Z)(a \leq b \Longrightarrow a + c \leq b + c)$
\end{ax}
\begin{ax}
  $(\forall a, b, c \in Z)(a \leq b e 0 \leq c \Longrightarrow a \cdot c \leq b \cdot c)$
\end{ax}

\begin{prop} % propriedade 3
  Seja $a \in Z$:\\
  1 - Se $a \leq 0$ então $a \geq 0$\\
  2 - Se $a \geq 0$ então $-a \leq 0$\\
  3 - $a^2 \leq 0$\\
  4 - $1 > 0$\\
\end{prop}

\section{Atividade 2}

Livro do Milies - página 18. Ebtrega 18/09.
\begin{equation} % FIXME need reset the number
  (-1) \cdot a = -a
\end{equation}
Se 
\begin{equation} % FIXME need reset the number
  a^2 = 0
\end{equation}
então $a = 0$

\section{Atividade 1}

Exercicio 3 e correlato a proposicao 3 das notas manuais

\nocite{*}
\bibliographystyle{ieeetr}
\bibliography{class_03_sept_13}

\end{document}