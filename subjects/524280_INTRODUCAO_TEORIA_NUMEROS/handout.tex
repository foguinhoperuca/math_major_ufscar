%% Copyright (C)  2013 Jefferson Campos.
%% Permission is granted to copy, distribute and/or modify this document
%% under the terms of the GNU Free Documentation License, Version 1.3
%% or any later version published by the Free Software Foundation;
%% with no Invariant Sections, no Front-Cover Texts, and no Back-Cover Texts.
%% A copy of the license is included in the section entitled "GNU
%% Free Documentation License".

\documentclass[a4paper,12pt]{article}
\usepackage[brazilian]{babel}
\usepackage[utf8]{inputenc}
\usepackage[T1]{fontenc}
\usepackage{hyperref}
\usepackage{url}
\usepackage{amssymb}
\usepackage{amsthm}
\usepackage{amsmath}
\usepackage[pdftex]{graphicx}
\newcommand{\HRule}{\rule{\linewidth}{0.5mm}}

% Números Naturais
\newtheorem{add_nat}{Adição}
\newtheorem{mult_nat}{Multiplicação}
\newtheorem{dist_nat}{Distributiva}
\newtheorem{prop_nat}{Proposição}
\newtheorem{definit_nat}{Definição}
\newtheorem{ax_nat}{Axioma}

% Números Inteiros
\newtheorem{add_int}{Adição}
\newtheorem{mult_int}{Multiplicação}
\newtheorem{dist_int}{Distributiva}
\newtheorem{prop_int}{Proposição}
\newtheorem{definit}{Definição}
\newtheorem{ax}{Axioma}

\begin{document}

\section{Aula 20-08-2013.}

Números Naturais.

\section{Aula 21-08-2013.}

Continuação números naturais.

\section{Aula 03-09-2013.}

O documento está, inicialmente dividido por aulas para facilitar a digitação. Posteriormente este material deve ser organizada de acordo com os teoremas, axiomas, etc, apresentados durante o curso.

% Primeira Parte da  Aula.
\subsection{Números Inteiros.}

Assume-se que os elementos do conjunto dos inteiro $\mathbb{Z}$ satisfazem os seguintes axiomas, $\forall a, b, c \in \mathbb{Z}$:

\begin{add_int}\label{A1}[Propriedade Associativa para Soma $+$]
  \begin{equation*}
    a + (b + c) = (a + b) + c
  \end{equation*}
\end{add_int}

\begin{add_int}\label{A2}[Existência do Elemento Neutro Aditivo]
  \begin{equation*}
    \exists |0 \in \mathbb{Z}| a + 0 = a = 0 + a
  \end{equation*}
\end{add_int}

\begin{add_int}\label{A3}[Existência do Oposto]
  \begin{equation*}
    \exists |-a \in \mathbb{Z}| a + (-a) = 0 = (-a) + a
  \end{equation*}
\end{add_int}

\begin{add_int}\label{A4}[Propriedade Comutativa]
  \begin{equation*}
    a + b = b + a
  \end{equation*}
\end{add_int}

\begin{mult_int}\label{M1}[Propriedade Associativa para ``$\cdot$'']
  \begin{equation*}
    a \cdot (b \cdot c) = (a \cdot b) \cdot c
  \end{equation*}
\end{mult_int}

\begin{mult_int}\label{M2}[Existência do Elemento Neutro Multiplicativo]
  \begin{equation*}
    \exists |1 \in \mathbb{Z}| 1 \cdot a = a = a \cdot 1
  \end{equation*}
\end{mult_int}

\begin{mult_int}\label{M3}[Propriedade Cancelativa]
  Sejam $\forall a, b, c \in \mathbb{Z}, a \cdot a \neq 0$\\
  Se $a \cdot b = a \cdot c$ então $b = c$
\end{mult_int}

\begin{mult_int}\label{M4}[Propriedade Comutativa]
  \begin{equation*}
    a \cdot b = b \cdot a
  \end{equation*}
\end{mult_int}

\begin{dist_int}\label{D1}[Propriedade Distributiva da Multiplicação com Relação à Adição]
  \begin{equation*}
    a \cdot (b + c) = a \cdot b + a \cdot c
  \end{equation*}
  \begin{equation*}
    (b + c) \cdot a = b \cdot a + c \cdot a
  \end{equation*}
\end{dist_int}

\begin{dist_int} % D2
  Propriedade distributiva da multiplicação com relação à adição:\\
  $a \cdot (b + c) = a \cdot b + a \cdot c$\\
  $(b + c) \cdot a = b \cdot a + c \cdot a$
\end{dist_int}

\begin{prop_int} % (1)
  Propriedade do cancelamento para $+$:\\
  $(\forall a, b, c \in \mathbb{Z})(a + b = a + c \Longrightarrow b = c)$\\
\end{prop_int}
\begin{proof}
    $\forall a, b, c \in \mathbb{Z}$ faz-se:% +(-a)
  \begin{equation*}
    a + b = a + c % A1
  \end{equation*}
  \begin{equation*}
    (-a) + (a + b) = (-a) + (a + c) % A3
  \end{equation*}
  \begin{equation*}
    ((-a) + a) + b = ((-a) + a) + c % A2
  \end{equation*}
  \begin{equation*}
    0 + b = 0 + c
  \end{equation*}
  \begin{equation*}
    b = c
  \end{equation*}
\end{proof}

\begin{prop_int} % (2)
  $(\forall a \in \mathbb{Z})(a \cdot 0 = 0 = 0 \cdot a)$
\end{prop_int}
\begin{proof}
  Assumindo
  \begin{equation*}
    a \cdot 0 = 0
  \end{equation*}
  % por em uma tabela
  $\forall a, b, c \in \mathbb{Z}$
  \begin{equation*}
   a \cdot (b + c) = a \cdot b + a \cdot c \tag{por Distributiva~\ref{D1}}   
  \end{equation*}
  tomando
  \begin{equation*}
    b = c = 0 
  \end{equation*}
  \begin{equation*}
    a (0 + 0) = a \cdot 0 + a \cdot 0 por (  a \cdot 0
  \end{equation*}
   % esta incompleto

  analogamento faz-se para
  \begin{equation*}
    \\
    0 = 0 \cdot a
  \end{equation*}
\end{proof}

\begin{prop_int} %(3)
  $(\forall a, b, \in Z)(a \cdot b = 0 \Longrightarrow a = 0$ ou $b = 0)$
\end{prop_int}

\begin{prop_int} %(4)
  Regra dos sinais:\\
  $(\forall a, b, \in Z)$ tem-se:\\
  \begin{equation} % FIXME need reset the number
    -(-a) = a
  \end{equation}
  \begin{equation} % FIXME need reset the number
    -(-a) \cdot b = -(a \cdot b) = a \cdot (-b)
  \end{equation}
  \begin{equation} % FIXME need reset the number
    (-a) \cdot (-b) = a \cdot b
  \end{equation}
\end{prop_int}

\subsection{Relação de Ordem.}

\begin{definit}
 Chama-se relação binária de $E$ em $F$  todo subconjunto $R$ de $E \times F$, isto é, sejam $E, F$ conjuntos e $E \times F = \{(m, n) | m \in E e n \in F\}$ assim, se $R$ é uma relação binária de $E$ em $F$, tem-se que $R$ está contido  $ E \times F$ % FIXME colocar simbolo do "C" está contido
\end{definit}

\begin{definit}
 precisa copiar...
\end{definit}

Assim a relação ``menor ou igual'', $\leq$, é uma relação de ordem parcial sobre $\mathbb{Z}$, sendo válido os axiomas:

\begin{ax}
  $(\forall  a \in Z)(a \leq a)$ reflexiva
\end{ax}
\begin{ax}
  $(\forall a, b \in Z)(a \leq b e b \leq a \Longrightarrow a = b)$ - anti-simétrica
\end{ax}
\begin{ax}
  $(\forall a, b, c \in Z)(a \leq b e b \leq c \Longrightarrow a \leq c)$ - transitiva
\end{ax}

Obs:

1 - $\mathbb{Z}$ é parcialmente ordenado mediante a relação $\leq$
2 - Se $a \leq b e a \neq b$ neste caso utiliza-se $a \leq b$
3 - Se $a \leq b$ logo $a - b \leq 0$
4 - Usam-se o símbolo $b \geq a ou b > a$

% ---

São válidos válidos, também os axiomas:

\begin{ax}
  Dados $a, b, \in Z$ tem-se que $a < b ou b < a ou a = b$ - tricotomia
\end{ax}

\begin{ax}
  $(\forall a, b, c \in Z)(a \leq b \Longrightarrow a + c \leq b + c)$
\end{ax}
\begin{ax}
  $(\forall a, b, c \in Z)(a \leq b e 0 \leq c \Longrightarrow a \cdot c \leq b \cdot c)$
\end{ax}

\begin{prop_int} % propriedade 3
  Seja $a \in Z$:\\
  1 - Se $a \leq 0$ então $a \geq 0$\\
  2 - Se $a \geq 0$ então $-a \leq 0$\\
  3 - $a^2 \leq 0$\\
  4 - $1 > 0$\\
\end{prop_int}

\section{Aula 04-09-2013.}

Assume-se que os elementos do conjunto dos inteiro $\mathbb{Z}$ satisfazem os seguintes axiomas, $\forall a, b, c \in Z$:

\begin{add_int} % D3
  Seja $a \in Z$. O valor absoluto (ou módulo) de $a$, $|aA$, é definido por:\\
  $$
  |x|=\left\{
    \begin{array}{rc}
      a,&\mbox{se}\quad a\ge 0,\\
      -a, &\mbox{se}\quad a<0.
    \end{array}\right.
  $$

obs:

1 - se $a \leq b$ e $b \leq c$, pode-se escrever $a \leq b \leq c$\\
2 - $\forall a \in Z, |a| \ge 0$\\

EX:

$|3| = 3$ pois $3 \geq 0$\\
$|-2| = -(-2) = 2$ pois $-2 \leq 0$
\end{add_int}

\begin{add_int} % Proposição 4
  $\forall a, b,c \in Z$\\

  1 - $|a| \ge 0 $ e $|a| = 0 \iff a = 0$\\
  2 - $-|a| \leq a |leq |a|$\\
  3 - $|-a| = |a|$\\
  4 - $|a \cdot b| = |a|\cdot|b|$\\
  5 - $|a| \leq b  \iff -b \leq a \leq b$\\
  6 - $|a + b| \leq |a| + |b|$ - desigualdade triangular\\
  7 - $ ||a| - |b|| \leq |a| - |b|$\\

  Sejam | parcialmente ordenados mediante a relação $\leq$ e $A $ está contido em $\mathbb{Z}$ com $A \neq \emptyset$
\end{add_int}

\begin{add_int} % Proposição 5
  1 - $L \in \mathbb{Z}$ é  um limite superior de $A  \iff (\forall x \in A \Longrightarrow x \leq L) $\\
  % 2 - $$\\ % copiar manualmente

Observação:

1 - se $L \in \mathbb{Z}$ é limite superior de A diz-se que A é limitado superior\\
2 - Se $l \in \mathbb{Z}$ é limite inferior de A diz-se que A é limitado inferior\\
\end{add_int}

\begin{add_int} % Definição
  1 - $M \in A$ é um máximo de $A  \iff (\forall x \in A \Longrightarrow x \leq M) $\\
  2 - $m \in A$ é um mpinimo de $A \iff (\forall x \in A \Longrightarrow x \geq M) $ \\

  Observação:
  
  $M = max(A) \in A$
  $m = mix(A) \in A$
\end{add_int}

\begin{add_int} % Proposição 5 - de novo?
  1 - Se $M = max(A)$ então $M$ é único.\\
  2 - $m  = min(A)$ então $,$ é único \\
\end{add_int}

O axioma abaixo pe ocnhecido como ``Princípio da Boa Ordem (POB)''. Todo conjunto não-vazio de inteiros não-negativos contém um mínimo.Em símbolos temos:\\
$A \subset \mathbb{Z}$ e $A \neq \emptyset$\\
Se $\forall x \in A, x \ge 0$ então existe $m = min(A)$

\begin{add_int} % Proposição 6
  Seja $a \in \mathbb{Z}$\\
  Se $0 \leq a \leq 1$ então\\
  $a = 0$ ou $a = 1$
\end{add_int}
\begin{proof}
  Estratégia de Demonstração: Reduction ad Absurdum\\

  1 - $v(p) =V$
  2 - Suponha por absurdo que $v(\neg q) = V (\neg q \iff \neg (q_1 \vee q_2) \iff \neg q_1 \wedge \neg q_2)$\\
  3 - Desenvolva a teoria a partir de 2\\
  4 - Obtenha uma proposição $r$ tal que $r \iff \neg p$ (absurdo!)
\end{proof}

\begin{add_int} % Proposição 7
  Propriedade de Arquimediana:\\
  Seja $A = \{x \in \mathbb{Z} | x \geq 0\}$\\
  Se $a, b \in A$ então\\
  $(\exists n \in A)(n \cdot a > b)$
\end{add_int}
\begin{proof}
  A estratégia utilizada é a redução ao absurdo.\\
  \begin{equation}
    Por hipótese é válido que a, b \in A (1)\\
    Suponha por absurdo que seja válido:\\
  \end{equation}
  \begin{equation}
    \neg q: (\forall n \in A)(n \cdot a \leq b) (2)\\
    Daí constrói-se o seguinte conjunto, considerando (2):\\
    n \cdot a \leq b \Longrightarrow 0 \leq b - n \cdot a\\
    assim temos:\\
  \end{equation}
  \begin{equation}
    S = \{b - n \cdot a | n \in A\} (3)\\
    Assim \forall x \in S tem-se x = b -n \cdot a \leq 0, n \in A. Daí pelo P.B.O:\\
  \end{equation}
  \begin{equation}
    \exists m = min(S) sendo m \in S (4)\\
    logo, \\
  \end{equation}
  \begin{equation}
    m = b = r \cdot a (5)\\
    para algum r \in A. considerando m' \in S | m' = b - (r + 1 ) \cdot a para algum r \in A. Assim temos:\\
  \end{equation}
  \begin{equation}
    m' = b - (r \cdot a + a)\\
    = b + (-r \cdot a - a)\\
    = (b - r \cdot a) - a = m - a\\
    \Longrightarrow m' = m - a
  \end{equation}

  Por (1) $a \in A$, assim $a > 0$\\
  $m - a < m$\\
  Daí\\
  $m' < m$ (6)\\

  Por (4) $m - min(S)$, logo pela definição 5 item 2\\
  $m \in S$ e $m\leq x \forall x in S$ (7)\\
  Por (6) e (7) e usando a transitividade de (O3) tem-se que:\\
  $m' < x, \forall x \in S$.\\
  Daí pela definição 5 item 2 temos que:\\
  $m' = min(S)$ (8)\\
  Tem se:\\
  $
  m = min(S)\\
  m' = min (S)
  $

  4th - Mas isso é um absurdo, pois o mínimo de S é único, pela proposição (5). Logo o absurdo foi considerar $n \cdot a < b, \forall a \in A$.\\
  Portanto, $(\exists n \in A)(n \cdot a > b)$
\end{proof}

% Proposição 7 Possui: PBO; definição 5 item 2; O3; Proposição 5; A base da prova é a prova por contradição.

\begin{add_int} % Proposição 8
  Seja $A \subset \mathbb{Z} \wedge A \neq \emptyset$\\
  Se $A$ é limitado inferiormente então $(\exists m \in A)(\forall x \in A \longrightarrow x \geq m)$
\end{add_int}
\begin{proof}
  Dica: $S = \{x - l | \forall x \in A\}$
\end{proof}

\section{Aula 10-09-2013.}

Unknow.... ????

\section{aula 11-09-2013.}

Hipótese de indução.

\section{PDF.}

Nesta seção estão os materiais dos PDF enviados.

% Aparentemente esta seção já foi copiada em sala de aula.

% % Página 13
% \subsection{Fundamentação Axiomática.}

% O conjunto dos números inteiros $\mathbb{Z}$ é construído a partir dos números naturais $\mathbb{N}$. Assim, temos:\\
% $\mathbb{N} = \{0, 1, 2, 3, ...\}$\\
% $\mathbb{Z} = \{..., -3, -2, -1, 0, 1, 2, 3, ... \}$\\
% $\mathbb{Z}^*  = \{..., -3, -2, -1, 1, 2, 3, ... \}$\\

% Analogamente ao que foi feito para $mathbb{N}$ define-se as operações sobre $mathbb{Z}$:\\

\section{Apêndice.}

\subsection{Técnicas de Provas.}

Para realizar as provas existem diversas técnicas que podem ser empregadas. Podemos citar:
\begin{itemize}
\item Demonstração direta;
\item Ad. Absurdum;
\item Contra-positiva $\neg q \rightarrow \neg p$;
\item Contra-exemplo;
\item Princípio da Indução Finita (PIF) ou Princípio da Indução Completa (PIC);
\end{itemize}

\section{Atividades.}

Esta seção contém as atividades propostas durante o curso. Estas atividades vão complementar o aprendizado proposto no texto principal e serão utilizadas como instrumentos de avaliação.

\subsection{Atividade 1}

Exercicio 3 e correlato a proposicao 3 das notas manuais

\subsection{Atividade 2}

Livro do Milies - página 18. Entrega 18/09.
\begin{equation} % FIXME need reset the number
  (-1) \cdot a = -a
\end{equation}
Se 
\begin{equation} % FIXME need reset the number
  a^2 = 0
\end{equation}
então $a = 0$

\nocite{*}
\bibliographystyle{ieeetr}
\bibliography{handout}

\end{document}