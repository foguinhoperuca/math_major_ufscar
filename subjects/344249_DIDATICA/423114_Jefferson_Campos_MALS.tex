\documentclass[a4paper,12pt]{article}
\usepackage[brazil]{babel}
\usepackage[utf8]{inputenc}
\usepackage[usenames]{color}
\usepackage[pdftex]{hyperref}

\title{Portifólio da Disciplina de Didática.}
\author{Jefferson Campos - RA 423114}
\date{\today}

\begin{document}

\maketitle

\newpage

\section{Narrativa de Algum Acontecimento da Vida Escolar.}

Um acontecimento marcante na minha vida escolar foi o meu professor de história (história do Brasil) quando eu fiz cursinho antes de eu cursar PD (processamento de dados) na FATEC/SO. As suas aulas eram bem dinâmicas e o assunto, que de certa forma inspira um ar mais tedioso, era absorvido de forma eficiente e principalmente de uma forma prazerosa. As suas aulas me influenciaram bastante e fez com que eu tomasse gosto pelo tema.\\
Outro fato marcante que é relevante, é que recentemente comecei o mestrado em ciências da computação na UFSCAR Sorocaba. Um dos meus coloegas (prof. Anderson) de classe foi o meu professor no curso técnico (durante o ensino médio) no ETE Fernando Prestes. O prof. Anderson foi uma grande influência, sendo que eram uma das aulas que eu mais gostava durante o curso. Hoje, acabei seguindo a profissão de analista de sistemas e tive o meu empurrão para a profissão a partir daí.

\section{Registro Sobre o Filme ``A Sociedade dos Poetas Mortos''.}

\subsection{Resumo.}

O filme mostra a atuação de um professor nada convencional, ainda mais no contexto social ao qual ele está inserido. Este professor, propositalmente, desafia o sistema vigente através de atuações diferenciadas, maneira de se portar entre outras caracteristicas. Esse desafio ao sistema vigente tem o intuito, não apenas da rebeldia sem fundamento e gratuita mas sim o intuito de mostrar uma nova visão sobre a vida e a sociedade como um todo além da visão tida como ``padrão''. Por seus métodos caracteristicos, esse professor chama atenção sobre si, o que gera um desconforto na comunidade acadêmica. Desconforto esse que resulta em represálias.

\subsection{Minha Opnião.}
No geral, gostei do filme. Salvo os exageros, comuns à todos os filme dramatizados, acho que o filme pode sim servir como inspiração para os professores.\\
O principal argumento do filme, na minha opnião, é a paixão pelo conteúdo. O prazer de aprender algo que seja de seu interesse o que leva à absorvição do conhecimento e não a sua simples memorização e mecanização. Isso, consequentemente, leva à critica do conhecimento absorvido e da realidade, por parte do aluno, tornando-o uma pessoa independente. Claro, como mostrado no filme (e hoje em dia, apesar da época não ser mais a mesma), isso se choca com uma sociedade mais conservadora, onde esse conhecimento novo e a independência criada por parte dos alunos gera um forte conflito com a ``sede de dominação e controle'' e a estrutura da sociedade.

\section{Reflexão da Discussão sobre a Vivência dos Professores em Sala de Aula.}

Pela discução e o depoimento dos alunos pude perceber que houveram algumas mudanças, superficias, na forma de ensino. No geral, o formato é o mesmo desde quando eu me formei (2003). Percebi, também, que os principais aspectos negativos são constantes como desinteresse dos alunos, a dificuldade de relacionamento entre professor e aluno e o relacionamento do aluno com a escola de forma geral entre outros.\\
Eu particularmente passei por escolas, que apesar de ter esses mesmos problemas, eu acredito que eles foram em um grau mais baixo. A frequência era menor. Não tive tanta experiência direta na escola com alunos com dificuldade de relacionamento, falta de recursos, entre outros. As instituições pelas quais passei fizeram um bom trabalho.\\
Entretanto, tive contato com alunos de outras escolas mais deficientes e periféricas. Depois, durante o meu curso de PD na FATEC/SO, participei de programas de inclusão digital também. Pude ver que algumas escolas eram bem sucateadas e que a qualidade era péssima.\\
Ao meu ver a escola piorou ou no máximo manteve a mesma qualidade. Isso é preocupante, pois vai afetar o futuro de todos.\\
Outro ponto interessante foi em relação as estratégias de ensino: especialmente a estratégia de se ``impor'' desde o primeiro dia, para adquirir o respeito do aluno e com o passar do tempo se tornar um professor mais ``amigo''. A estratégia contrária eu senti na pele durante a minha vivência como professor e aprendi que ela não é a mais adequada, na maioria das vezes.\\


\section{Reflexão sobre o Currículo Escolar.}

\subsection{Pontos Positivos.}

\subsubsection{Interdisciplinaridade.}

É interessante trabalhar um conteúdo de forma que ele faça uma ponte entre outros conhecimentos e que por sua vez faça uma ponte com a realidade do aluno. Assim o aprendizado se torna mais atrativo e os alunos vão motivar-se a aprofundar-se no estudo.

\subsubsection{Tecnologia.}

Uso de tecnologias (especialmente de tecnologia da informação) no apoio ao ensino. As aulas tornam-se mais dinâmicas pois o uso do computador pode facilitar a visualização de conceitos. Particularmente a geometria é uma área beneficiada, sendo mais simples o ensino onde o aluno pode visualizar e interagir com o objeto matemático.

\subsubsection{Computação.}

Outro ponto, que apesar de não estar explicitamente citado, mas que seriam interessante ser apresentado (provavelmente no ensino médio) é o ensino básico  de computação. Esse ensino de computação não abrange o uso de computadores em si mas sim o desenvolvimento de programas que irão manipular dados. Acredito que isso contribua para que a matemática seja vista de uma forma mais prática.

\subsection{Pontos negativos.}

Professor decidir quando e como abordar a geometria: a falta de planejamento ou a troca constante de professores (ou outros problemas adminitrativos) podem interfirir no aprendizado dos alunos.

\section{Reflexão sobre o Planejamento Escolar (Seminários).}

O empréstimo de conceitos e modelos de uma área pode ser útil, uma vez que ambas as áres tenham problemas que compartilhem de características comuns. Entretanto, é necessário realizar uma crítica, uma reflexão sobre esses modelos ``emprestados'' pois, apesar das semelhanças, os problemas de duas áreas distintas podem não ser exatamente os mesmos. \\
A grande inspiração das propostas de planejamento escolar, como apresentada no texto de referência, é o modelo de gestão mais tradicional como a administração científica. \\
Entretanto, o texto peca em alguns aspectos justamente na hora de aplicar esses conceitos da admnistração científica na gestão escolar. Um exemplo é a ``assimilação'' do conteúdo pelos alunos. A falha está em justamente presumir que os alunos apenas vão ``absorver'' ou ``consumir'' o conhecimento. Seria melhor tratar como se os alunos apropriassem-se do conhecimento, no sentido de receber uma informação nova e, ao invés de simplesmente gravar, o aluno faz uma crítica o que pode levar à uma modificação, uma adaptação desse conhecimento antes que ele seja de fato incorporado ao seu ``portifólio''.

\section{Análise do Plano de Ensino - Matemática.}

O plano de aula escolhido foi o do \href{http://revistaescola.abril.com.br/ensino-medio/plano-aula-planejamento-financeiro-calculo-juros-646550.shtml}{planejamento financeiro e cálculo de juros (link para documento online aqui)}.


\subsection{Indícios de Abordagem Pedagógica Assumida.}

A abordagem pedagógica assumida é a subordinação da teoria à prática, característica do século XX em diante. Essa abordagem é feita, partindo da introdução da teoria e um debate sobre a mesma mas com o foco na prática posterior e na construção do conhecimento por parte do aluno. A dicussão e participação do aluno é vital para o suscesso da aula, sendo que o aluno irá, a partir da própria experiência, construir uma o seu próprio planejamento financeiro, levando em conta os pontos expostos pelo professor e pela discussão anterior.

\subsection{Relação Professor e Aluno.}

A relação aluno-professor será onde o professor será um mediador e um guia, expondo os principais tópicos sobre o assunto e conduzirá as dicussões. Já o aluno, cabe a participação da discussão, contribuindo com eperiências próprias e também na construção do planejamento financeiro pessoal.

\subsection{Objetivos.}
O principal objetivo é fazer com que o aluno possa refletir sobre o dinheiro e seu uso, dando enfase ao uso racional do mesmo.
\begin{itemize}
\item Mostrar a importância do planejamento financeiro aos alunos;
\item Efetuar cálculos com foco na importância de economizar, gastar apenas aquilo que é necessário. - Empregar nos cálculos juros simples e composto;
\item Construir diversos tipos de planilhas de gastos (utilizando planilhas eletrônicas);
\end{itemize}

\subsection{Estratégias de Ensino.}

As estratégias de ensino adotadas privilegiam a prática, apoiada por uma teoria que ajuda a fundamentá-la. As principais estratégias são:
\begin{itemize}
\item Exposição dos principais tópicos do tema aos alunos, tendo em mente a sua aplicação;
\item Dicussão sobre os tópicos apresentados. Contribuição dos alunos com suas próprias experiências;
\item Prática da teoria exposta anteriormente através da implementação de um planejamento financeiro pessoal para cada aluno baseado em sua realidade. O refinamento será dado através das discussões em sala de aula;
\item Uso, preferencialmente, de uma planilha eletrônica evidenciando a vantagem de uso da mesma.
\end{itemize}

\subsection{Avaliação.}

A avaliação do aluno deve ser feita com base na participação dos mesmos. A participação é fundamental pois há a necessidade de uma reflexão do aluno para que ele possa propor novas/melhores formas de gerenciar o dinheiro e que ele seja capaz de identificar boas e más práticas de gestão. Especificamente, o aluno pode ser avaliado nos seguintes quesitos:
\begin{itemize}
\item Os alunos são capazes de identificar gastos recorrentes?
\item Os alunos são capazes de identificar qual o melhor investimento, levando em consideração o prazo, renda e juros?
\item Os alunos são capazes de determinar o que pode ser investido e o que pode ser poupado?
\end{itemize}

\subsection{Destacar as Considerações sobre Possibilidades de Desenvolvimento da Proposta.}

A proposta pode ser desenvolvida através de uma discussão sobre a importância do controle financeiro e a forma de consumo e as metas de consumo/desejos. Também deve ser discutido os juros e as formas e prazos de pagamento. Sobre tudo, o aluno deve ter uma visão da melhor prática de acordo com a situação.\\
A proposta pode ser desenvolvida, também, com uma atividade prática de colocar os gastos ``no papel'' e fazer uma análise da qualidade dos mesmos. Essa prática pode ser feita diretamente em um papel ou mesmo em um computador com uma planilha eletrônica.

\section{Avaliação Geral da Disciplina.}

Estou satisfeito com a disciplina e com o transcorrer das aulas.
Achei particularmente interessante os debates ocorridos, que me aproximou um pouco da realidade escolar. Eu sou formado e trabalho na área de tecnologia da informação, portanto o meu contato com a área educacional não é tão grande.
Fiquei decepicionado, confesso, ao ver que as condições de ensino e as dificuldades relatadas pelos colegas que atuam na rede pública persistem desde a época em que eu era estudante também (10 anos atrás). Entretanto, consigo ver o valor que a disciplina tem para tentar reverter as condições atuais, especialmente elucidando a dinâmica da escola e analisando as formas de ``montar'' aulas mais interessantes e atrativas que quebrem com a monotonia da escola e tornem as aulas realmente interessante, instigando os alunos a se aprofundarem nos assuntos abordados.
Outro ponto relevante na disciplina foi a análise do funcionamento do dia a dia da escola e o planejamento escolar. Este tópicos me deram uma visão diferente da qual eu tinha, apenas como aluno. A visão macro do funcionamento , do planejamento da escola e das aulas é extremamente importante para o professor. É  uma ferramenta excelente para a reflexão e melhoria da qualidade de suas aulas.
Finalmente, a análise do plano de aula me deu uma visão mais clara de como preparar uma aula. Aparentemente óbvio porém este tópico me fez atentar ao fato que uma aula precisa ter um plano claro, com objetivos e meios de atingi-los. O plano de aula é fundamental pra isso.


\end{document}
