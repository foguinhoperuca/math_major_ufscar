\documentclass[a4paper,12pt]{article}
\usepackage[brazil]{babel}
\usepackage[latin]{inputec}
\begin{document}

\section{= Partindo do Exemplo da Aula Passada.}
$m_1 = 23,6g$ \Longrightarrow 2,36
$m_2 = 0,84g$ \Longrightarrow 8,4

\section{= Arredondamento de numero.}
Devido a respota de a uma operação aritimetica que contem mais algarismo que os significativos:
\begin{itemsize}
\item Quando o algarismo seguinte ao ultio numero a ser mantido e menor que 5, todos os algarismos indesejaveis devem ser descartados e o ultimo numero eh mantido intacto. 
  \begin{enumerate}
    \item Ex01.: $2,14$ \longrightarrow $2,1$
    \item Ex.02: $4,372$ \longrightarrow $4,37$
  \end{enumerate}
\item Quando o algarismo seguinte ao numero a ser mantido \grave{e} maior que $5$ ou $5$ seguido de outro digito, o ultimo numro e aumentado de 1 e os algarismos indesejaveis sao descartaos.
  \begin{enumerate}
    \item Ex01.: $7,5647$ \longrightarrow $7,565$
    \item Ex.02: $3,5501$ \longrightarrow $3,6$
  \end{enumerate}
\item Quando o algarismo seguinte ao ultimo numero a ser mantido \grave{e} um $5$ (seco!) ou um $5$ seguido de zeros.
  \begin{enumerate}
    \item Se o \grave{u}ltimo algarismo a se mantido for \text{impar}, ele \grave{e} aumentado de $1$ e o $5$ indesejavel (e eventuais $0$) s\tilde{a}o descartado
    \begin{itemsize}
      \item Ex.: $7,635 = 7,64$
    \end{itemsize}
    \item Se o \grave{u}ltimo algarismo a se mantido for \textit{par} ele eh mantido inalterado e o 5 indesejavel (e eventuais 0's) \grave{e} descartado
      \begin{itemsize}
        \item Ex.: $3,250$ \longrightarrow $3,2$
        \item Ex.: $8,105 = 8,10$
      \end{itemsize}
  \end{enumerate}
\end{itemsize}

\subsection {Adicao & Subtacrao.}
$6,3 + 2,14 = 8,44$
Prevalece o menor numero de casas \longrightarrow $90 - 2,14 = 8785$
\subsection{Multiplicacao e Diviao}
$6,3 * 2,14 = 13,482 = 13$
\frac{$6,3$}{$2,14$} = $2,4439254 = 2,9$

\textit{Prevalece o valor com o menor numero de algarismos significativos}

\frac{$(8,728 - 4,3)$}{$9,27$} = \frac{$4,428$}{$9,27$} = $0,4776$ = $0,47$

\textit{\textbf{Sempre arredondar no final da conta. Se arredondar no meio, vai dar diferença (os erros de precisao vao se acumulando).}}

\section{== Parte 2 - Experimento}
Temperatura \grave{e} a medida de calor ou frieza de um objeto. De fato, a temperatura determina a direção do fluxo de calor. O valor sempre flui expontaneamente de uma subst\circ{a}ncia T+\uparrow para outra T+\downarrow.

$K = C + 273,15$

$C = \frac{5}{9}(F - 32)$
ou
$F = \frac{9}{5}(C + 32)$

Fazer a relacao entre F x K x C

\subsection{Lei $0$ da termodinamica}


1 - Pq o menisco do mercurio eh para cima e o menisco do alcool eh para baixo?
2 - Pq usar o mercurio para medir a temperatura do corpo e nao outra substancia?

\end{document}
