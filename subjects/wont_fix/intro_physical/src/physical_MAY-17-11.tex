\documentclass[a4paper,12pt]{article}
\usepackage[brazil]{babel}
\usepackage[latin1]{inputenc}
\begin{document}
\section{General.}
Lecture - Forcas.
\section{Forcas Sobre o Bloco}
2 blocos no plano reto \to sem atrito.

Aplicar a 2nd lei de newton nos blocos.
a_1x = a_2x = a

Bloco 1:
Horizontal \to 2F_x = m_1 * a_1x
F_2_1 = m_1 * a (1)

2nd Lei no Bloco:

\uparrow y

2F_b - m * a
2F_b = m * 0
2F_b = 0

+n - f_b = 0

f_b = n

\subsection{Na vertical.}
2F_y = m_1 * a_1
N_1 - P_1 = 0
N_1 = P_1
N_1 = m_1 * a_1 (2)

No bloco 2:
Horizontal \to 2 * F_x = m_2 * a_2_x

F - F_12 = m_2 * a

Na vertical:
2 * F_y = M_2 * a_2_y

N_2 - P_2 = 0

N_2 = P_2

M_2 = M_2_g

Vetorialmente:
F_2_1 = -F_2_1

F_2_1 = F_1_2 = T (5)

Substituindo (5) em (1 e (3)

F_2_1 = m_1 * a

T = m_1 * a (6)

F - F_1_2 = M_2 * a

F - T = M_2 * a (7)

Siupondo que sao dadas  forca F e as massas m_1 e M_2 e e pedido para calcular T e a.

Substituindo (6) em (7):

F - m_1 = m_2 * a
F = m_1 * a + m_2a

a = \frac{F}{m_1 + m_2}

Substituindo a em (6)

T = m_1 * \frac{F}{m_1 + m_2}

T = \frac{m_1 * F}{m_1 + m_2}

\subsection{Plano Inclinado}

Utilizar o teorema de pitagoras e decompor a forca P em P_x e P_y (lei de soma dos vetores)


\end{document}
