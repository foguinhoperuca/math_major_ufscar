\begin{quote}
    Copyright \copyright{}  2011  Jefferson Campos - foguinho.peruca@gmail.com.
    Permission is granted to copy, distribute and/or modify this document
    under the terms of the GNU Free Documentation License, Version 1.3
    or any later version published by the Free Software Foundation;
    with no Invariant Sections, no Front-Cover Texts, and no Back-Cover Texts.
    A copy of the license is included in the section entitled ``GNU
    Free Documentation License''.
\end{quote}

\documentclass[a4paper,12pt]{article}
\usepackage[brazil]{babel}
\usepackage[latin1]{inputec}
\begin{document}

\section Revis\~ao.

\begin{enumeration}
\item Dada a fun\c{c}\~ao, d\^e o dom\'inio:
\begin{itemize}
\item f(x) = {2x, se x <= 1} ou {-x + 1, se x > 1}
\item f(x) = {4x^2, se 0 < x <= 4} ou {-1 se x < 0} ou {x-5 se x >= 4}
\end{itemize}
\item De o dom\'inio de f(x):
\begin{itemize}
\item
\item
\end{itemize}
\item Sejam f(x)
\begin{itemize}
\item
\item
\end{itemize}
\end{enumeration}








%% \section Funcao Quadratica ou do 2nd grau

%% \subsection Lei da funcao (formula)

%% \newtheorem[functionLaw]{Como Obter uma imagem?}
%% \begin{functionLaw}
%% $f(x) = ax² + bx + c$

%% onde:

%% a \to coeficiente que determina o sinal da funcao

%% b \to ?

%% c \to ?
%% \end{functionLaw}


%% \subsection Grapfico: Parabola.

%% Sao 6 possibilidades

%% \begin{enumerate}
%% \item concavidade voltada para cima, cortando o eixo x.
%% \item concavidade voltada para cima, tangeciando o x (toca somente em um unico ponto)
%% \item concavidade voltada para cima, onde nao toca o eixo x.
%% \item concavidade voltada para baixo, cortando o eixo x.
%% \item concavidade voltada para baixo, tangeciando o x (toca somente em um unico ponto)
%% \item concavidade voltada para baixo, onde nao toca o eixo x.
%% \end{enumerate}

%% GENERAL:
%% \begin{itemize}
%% \item Toda parabola cruza o eixo y quando x = 0. Logo, o ponto de cruzamento é da forma $(0, c)$ \to valor de $y$
%% \item O vertice delimita o crescimento ou decrescimento da parabola.
%% \item Sempre que houver apenas 1 conjunto de 2 pontos, é necessario utilizar utilizar um outro conjunto de 2 pontos.
%% \end{itemize}


%% \subsection Grafico 1.
%% \begin{itemize}
%% \item Concavidade para cima: $a > 0$
%% \item Há 2 cruzamentos com o eixo x $(x_1,0)$ e $(x_2, 0)$
%% \item Os valores $x_1$ e $x_2$ sao as raizes ou zeros de funcao para $ax^2+bx+c=0$, onde o $0$ será o valor de $y$.
%% \item Discriminante de $\delta = b^2-4ac$ é positivo
%% \item O vertice $(x, y)$ é um ponto de mínimo, logo o conjunto image é dado por $\Im(f(x)) = { y \in \Re / y \ge y_v}$
%% \end{itemize}

%% Exemplo:
%% $
%% f(x)=x^2-2x
%% a = 1
%% b = -2
%% c = 0
%% $

%% Raizes
%% $
%% y = 0
%% x^2-2x=0
%% \delta = (b)^2-4ac
%% \delta = (-2)^2-4*1*0
%% \delta = 4 - 0
%% delta = 0
%% x = \frac{-b\pm\sqrt{\delta}}{2a}
%% $

%% Logo, os pontos de cruzamento com o eixo x são $(0,0) e (2,0)$

%% Vertices
%% $x_y=\frac{-b}{2a}$ \to $x_y=\frac{-(-2)}{2(1)}$ \to $1$
%% $y_x=\frac{-\delta}{4a}$ \to $y_x=\frac{-(4)}{4(1)}$ \to $-1$

%% \Im (f(x)) = {y \in \Re / y \ge -1 }

%% O cruzamento com o eixo y é dado pelo ponto (0, 0) \to c


%% \subsection Grafico 2.
%% \begin{itemize}
%% \item Concavidade para cima: $a > 0$
%% \item Há 2 cruzamentos com o eixo, sendo \delta = 0. Portanto, há somente 1 raiz.
%% \item é a propria raiz. $\Im (f(x)) = { y \in \Re / y \ge 0 } ou \Re $
%% \end{itemize}

%% Exemplo:
%% $f(x) = x^2+2x+1$
%% sendo:
%% $
%% a = 1 \to concavidade para cima
%% b = 2
%% c = 1
%% $

%% Vétice e Raiz (-1, 0)

%% $
%% x^2+2x+1=0
%% \delta = (2)^2 - 4(1)(1) = 4 - 4 = 0
%% x = \frac{-(2)\pm \sqrt{0}}{2(1)} = -1
%% $

%% Cruzamento com o eixo y: (0, 1) \to c




%% \section TODO - HOMEWORK
%% Elaborar a caracteristicas e exemlos dos demais esbocos das parabolas.

\end{document}
