\documentclass[a4paper,12pt]{article}
\usepackage[brazil]{babel}
\usepackage[latin1]{inputenc}
\begin{document}


= Fundamentos Matemática - MAR-28-11.

Faltou fazer o exercício proposto na aula passada! Bad bad student...

== Exemplo 5 - apostila.
y = f(x) = 1 / x

Dm(f(x)) = ]-infinito, 0[ U ]0, +infinito[
ou IR - {0]

== Exemplo 4.
y = f(x) = sqrt(2,(x^2 - 1))

=== Lembrando.
Não existe raiz negativa real de índice par (raiz)

== Restrição de Domínio.
x² - 1 >= 0 (Condição de existência)

Dom(f(x)) = ]-\infty, -1] U [1, +\infty[
Im(f(x)) = [0, +\infty[ || {y \in IR / y >= 0}

----------(o)----(o)----------
  	  -1      1

=== Comentário a respeito de lógica.
TRUE || FALSE = TRUE

=== Como podemos resolver x² - 1 > 0
1) Resolver a equação x² - 1 = 0;
x² = 1 ---> 
x = +-sqrt(1) --->
x = +1; x = -1; (raízes)

2) Precisamos calcular valores de "x" \ x² -1 >= 0

   	      	 x² - 1 = 0
   	      	  ^
   	      	  |
   	      	  |
-----------I------I-----------
	  -1      1
	   |
	   |
 	   v
      x² - 1 = 0

\forall x < -1 ; x² - 1 > 0
\forall x > 1; x² -1 > 0


==== Marmita cultural
Qual o significado da srqt?
     Geometricamente, é o lado de um quadrado de área 1.
     Posso dizer ainda que a sqrt(2, 1) = 1. Geometricamente, não faz sentido sqrt(2, 1) = -1 (experimenta desenhar um quadrado de lado -1...)

==== Equação e Inequação.
Lembre-se, equação, apenas, pode calcular uma raíz. Inequação, ao contrário, não pode calcular uma raíz, apenas dá informação do sinal. Obviamente, ambos se complementam.


== Pag 41.
=== Grph Funcção.
y = f(x)
G(f) means gráfico da função f.
G(f) = {(x, f(x) / x \in Dom(f)}

todo elemento de 'x' deve corresponder com um único elemento de 'y'

=== Exemplo 2.3
Não pode ligar os pontos. Pq? pois não é uma curva, visto que os elementos 'dia' são números naturais e portanto, não possuem infinitos números entre si. Por exemplo, entre 1 e 2.

Qual o domíno dessa função?
Dom(f(x)) = { x \in IN / 1 <= x <= 7}

==== Definições.
abscissa --> x
ordenadas --> y

\end{document}
