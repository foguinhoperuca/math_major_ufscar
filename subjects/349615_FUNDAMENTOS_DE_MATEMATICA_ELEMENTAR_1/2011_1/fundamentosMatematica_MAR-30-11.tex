\documentclass[a4paper,12pt]{article}
\usepackage[brazil]{babel}
\usepackage[latin1]{inputenc}
\begin{document}

\section{= Fundamentos.}
% Nao esquecer da calculadora fisica (not mobile)
% Nao esquecer do papel quadriculado.
% Verificar software matematicos open-source - geogebra e winplot (alternativas linux tb)

\section{== Graficos}
\subsection{=== p41}
g=f(x)

p 41

Dom(f) = {x $\in \aleph$ / x = 1, ..., 7}
\Im = {y $\in \aleph$ / y = 360, 497, 510, 870, 950}

Diagrama de flechas:

1 --> 360
2 --> 497
3 --> 510
4 --> 870
5 --> 950
6 --> (repete) 497
7 --> (repete) 510

x -->  y = f(x)

Pq nao pode 'ligar' os pontos do grph?
  Nao pode ser um alinhamento de pontos pois a variacao nao e constante.

\subsection{=== p43}
[4]
Seja f(x) = $\frac{1}{x}$

Construa o grph

OBS: Essa funcao eh espelhada nos quadrantes 1 e 3

Quest�es extras.
1) O que ocorre com o comportamento de $x > 0$
  Ele se aproxima, cada vez mais do eixo x (aproxima-se de 0). Sempre tende a 0 (mas nunca e 0).
Paulo: Para $x > 0$,  y aproxima-se cada vez mais do valor 0.
2) E no caso de $x < 0$, o que ocorre com y?
  Tb, ele se aproxima cada vez mais do eixo x (aproxima-se de 0). Sempre tende a 0 (mas nunca e 0).
Paulo: Para $x < 0$, y tende a zero a partir de valores positivos.
3) Para x muito proximo de 0, qual o comportamento de y?
  Ele se aproxima do eixo y.
Paulo: Se x tiver proximo de zero, a partir de numeros positivos, y tende ao + $\propto$. Se x estiver proximo de zero a partir de numeros negativos, y tende ao - $\propto$

Em todos os casos, ele nuca toca nenhum eixo...


\end{document}
