\documentclass[a4paper,12pt]{article}
\usepackage[brazil]{babel}
\usepackage[latin1]{inputec}
\begin{document}

\section Que numero excede o seu quadrado?

$x>x^2$ é possível?
Para tornar uma função é necessário ter um igualdade. portanto, posso fazer x-x^2>0 e transformando em uma igualdade: x-x^2=0. Agora sim, faz sentido.

\subsection Outra ideia:
Para $x >= 1$ temos $x <= x^2$
Para$ x<=1$ temos $x <= x^2$
Para $1 < x < 0$ temos $x < x ^2$
O intervalo válido de x é [0, 1]
Logo, f(x) = x - x ^ 2 onde f(x) >= 0

Domínio: [0, 1]
Imagem: f(x) >= 0

\subject Exercicio 5 item c.
$f(x)=x^2+5x-6$
Dom f(x) = \Re

\newtheorem[ImgQuadrica]{Conjunto Imagem de uma Função Quádrica.}
\begin{Imgquadrica}
O conjunto imagem é determinado a partir de $y=\frac{-\delta}{4a}$ onde $\delta = b^2-4ac$.
\end{Imgquadrica}

$\delta = -\frac{1}{4}$ (menor valor para x)


\newtheorem[ObterImagem]{Como Obter uma imagem?}
\begin{ObterImagem}
Para $x \neq 3$ temos que pensar em $x \gt 3$ ou $x \lt 3$
\begin{tabular}{|c||c|}
\hline
\multicolum{2}{|c}{\textbf{Valores de X e Y}}\\
\hline
&X &Y\\
4& 4&\\
40& 1&\\
400& 100&\\


Como obter $\Im$?
O conjunto imagem é determinado a partir de $y=\frac{-\delta}{4a}$ onde $\delta = b^2-4ac$.
\end{Imgquadrica}


\end{document}

Dia 02/5 - prova de fundamentos
Dia 03/05 - prova biologia.
