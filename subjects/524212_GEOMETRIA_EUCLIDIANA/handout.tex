%% Copyright (C)  2013 Jefferson Campos.
%% Permission is granted to copy, distribute and/or modify this document
%% under the terms of the GNU Free Documentation License, Version 1.3
%% or any later version published by the Free Software Foundation;
%% with no Invariant Sections, no Front-Cover Texts, and no Back-Cover Texts.
%% A copy of the license is included in the section entitled "GNU
%% Free Documentation License".

\documentclass[a4paper,12pt]{article}
\usepackage[brazilian]{babel}
\usepackage[utf8]{inputenc}
\usepackage[T1]{fontenc}
\usepackage{hyperref}
\usepackage{url}
\usepackage{amssymb}
\usepackage{amsthm}
\usepackage{amsmath}
\usepackage[pdftex]{graphicx}
\newcommand{\HRule}{\rule{\linewidth}{0.5mm}}

% Números Inteiros
\newtheorem{prop}{Proposição}
\newtheorem{definit}{Definição}
\newtheorem{ax}{Axioma}

\begin{document}

% O documento está, inicialmente dividido por aulas para facilitar a digitação. Posteriormente este material deve ser organizada de acordo com os teoremas, axiomas, etc, apresentados durante o curso.

\section{Aula 11-09-2013.}

\section{Atividades.}

Esta seção contém as atividades propostas durante o curso. Estas atividades vão complementar o aprendizado proposto no texto principal e serão utilizadas como instrumentos de avaliação.

\subsection{Atividade ?}

...

\section{Apêndice.}

\subsection{Técnicas de Provas.}

Para realizar as provas existem diversas técnicas que podem ser empregadas. Podemos citar:
\begin{itemize}
\item Demonstração direta;
\item Ad. Absurdum;
\item Contra-positiva $\neg q \rightarrow \neg p$;
\item Contra-exemplo;
\item Princípio da Indução Finita (PIF) ou Princípio da Indução Completa (PIC);
\end{itemize}

\nocite{*}
\bibliographystyle{ieeetr}
\bibliography{handout}

\end{document}