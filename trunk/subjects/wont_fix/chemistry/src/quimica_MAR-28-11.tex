\documentclass[a4paper,12pt]{article}
\usepackage[brazil]{babel}
\usepackage[latin1]{inputenc}
\begin{document}


\section{Quimica.}
Pegar o livro recomendado pelo professor?

\section{== Unidades de medida.}
Muitas das propriedades da matéria são quntitativas, isto é, associadas a números. Quando um número representa uma medida, 

Ex.: Comprimento de um lápis.
R = 17.5 --> não diz nada.
R = 17.5 cm --> descreve adequadamente o seu comprimento.

\section{=== Unidades.}
  * Unidades usadas em medidas científicas.
  * Sistema metrico
    * Franca, no final do seculo XVIII
    * adotado internacionalmente a partir de 1960.
    * Systeme Intenational d'Unitas

\section{=== Unidades do Sistema Internacional.}
São 7 unidades básicas:

\begin{enumerate}
\item Grandeza Fisica		Nome da Unidade		Abreviatura
\item Massa    			Quilograma		Kg
\item Comprimento		Metro			m
\item Tempo			segundo			s
\item Temperatura		Kelvin			K
\item Quantidade de materia	Mol			mol
\item Corrente eletrica		Ampére			A
\item Intensidade luminosa	Candela			cd
\end {enumerate}

Ex.:
v = d / t --> usando as medidas seria m / s (unidade derivadas da unidade básica)

metro ==> metrons - que mede --> décima milionésima parte de distância do polo norte do equador, no meridiano que passa por Paris.

\section{==== Aplicações}

Qual a distância entre o Sol e Plutão.
6 000 000 000 000 m

Qual a massa média de ua asa de abelha?
0, 000 000 050 Kg

\section{==== Dificuldades de trabalhar com esses digitos?}
Nota{\ccedil}{\tilde a}o cientifica \Rightarrow Ax10^n

A = nr contendo um unico digito diferente de zero e a esquerda da virgula.
nr \Rightarrow numero inteiro


\section{=== Prefixos Usados no Sistema Internacional.}
Ver pag. 13 - livro do Braum - tab 15

\begin{enumerate}
\item Prefixo	                Abreviatura	        Significado		Exemplo
\item Giga    			G		        10⁹ (10 ^ 9)            1 gigametro = 1x10⁹ (10 ^ 9)
\item Mega		        M			10⁶ (10 ^ 6)            1 megamêtro = 1x10⁶ (10 ^ 6)
\item Kilo			K			10³ (10 ^ 3)            1 kilometro = 1x10³ (10 ^ 3)
\end {enumerate}

\begin {enumerate}
\item Prefixo	                Abreviatura	        Significado		Exemplo
\item Mili    			m		        10⁻³ (10 ^ -3)          1 milímetro = 1x10⁻³ (10 ^ -3)
\item Micro		        \mu			10⁻⁶ (10 ^ -6)          1 micrometro = 1x10⁻⁶ (10 ^ -6)
\item Nano			n			10⁻⁹ (10 ^ -9)          1 nanometro = 1x10⁻⁹ (10 ^ -9)
\item Pico		        P			10⁻¹² (10 ^ -12)        1 picometro = 1x10⁻¹² (10 ^ -12)
\item Femte			f			10⁻¹⁵ (10 ^ -15)        1 femtemetro(?) = 1x10⁻¹⁵ (10 ^ -15)
\end {enumerate}

\section{=== Examples}
55473Kg \Rightarrow 5,5473 x 10³ Kg
000134 \Rightarrow 1,34 x 10³ s .:. 1.34ms
17248 \Rightarrow 17248 x 10³L


\end {document}
