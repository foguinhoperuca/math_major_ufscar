\documentclass[a4paper,12pt]{article}
\usepackage[brazil]{babel}
\usepackage[latin1]{inputenc}
\begin{document}

\section{== Incerteza na Medida}

Numeros exatos = valores conhecidos com exatidao
Numeros inexatos = valores tem alguma incerteza
Ex:
  Existem exatamente 12 ovos em uma duzia.
  100g em 1Kg
  2,54cm em uma polegada.

Erro sistematico = falta no equipamento ou no projeto do equipamento.
  Ex. medidor de pH.

Erro Aleatorio = provem de efeitos de variaveis que nao estao controladas nas medidas. Ex. ruidos nos equipamentos; pessoas diferentes fazendo as medidas, etc.

Precisao = medida do grau de aproximacao entre os valores das medidas individuais.
Exatidao = indica qual grau de aproximacao entre individuais e o valor correto ou verdadeiro.

\section{== Algarismos Significativos.}

A magnitude de uma grandeza fisica e determinda por meio de instrumentos de medida apropriado.
Ex.
  massa de um objeto -- balanca
  comprimento de uma parede -- trena

Ex. Medidas de laboratorio -- balanca analitica e semi-analitica.

COnsequencia = A maginitude de uma gradeza pode ter valores mais ou menos preciso.
Precisao e regletida diretamente pelo n de casas decimais que o valor da grandeza contem.
Ex.
  Objetos pesados com balancas diferentes.
$\m_1 = 23,6g$ (precisao com decimo de gramas - uma casa apos a virgula)
$m_2 = 0,84g$ (precisao com centesimo de gramas - duas casas apos a virgula)

Os digitos que aparece no valor de uma grandeza usando notacao cientifica sao denomindados \textit{algarismos significativos}
Ex
  2,2 - 2 algarismos
  2,2405- cinco algarismos.

Conclusao: Quanto maior os algarismos significativos maior e a certeza evolvida na medida.


Obs.: Em qualquer medida relatada aproximadamentem todos os digitosdiferentes de zero sao significativo.
  Zeros podem se usados como parte do valor medido ou para alocar a virgila.
Obs.2: pode ser ou nao significativos dependendo de como aparecerem

\begin{enumerate}
\item Zero entre digitos diferentes de zero sao sempre significativos
  Ex 1.03xm
\item Zeros no inicio de um numero nunca sao significativos
  Ex 0,02g (1AS)
     0,026cm (2AS)

\item Zero no final de um numero e apos a virgula sempre significaticos.
Ex 0,0200g (3AS)
3,0cm (2AES)

\item Quando numero termina em zero mais nao contem virgula, zeros poden nao ser significativos
Ex:
  130cm - 
  10300g - 1,03 * 10⁴(3)
           1,030 * 10⁴ (4)
           1,0300 * 10⁵ (5)

\item Zero no final de um número e apos a virgula sao sempre significativos
Ex   0,0200g (3 AS 5)

Voltar ao exemplo:
$m_1 = 23,6cm$
$m_2 = 0,84g$ (8,4 * 10⁻¹)? ou 8,40 * 10⁻¹?
\end{enumerate}


\end{document}
