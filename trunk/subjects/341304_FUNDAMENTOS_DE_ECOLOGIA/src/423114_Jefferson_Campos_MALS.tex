\documentclass[a4paper,12pt]{article}
\usepackage[brazil]{babel}
\usepackage[utf8]{inputenc}
\usepackage[usenames]{color}
\usepackage[version=3]{mhchem}
\title{Estudo Dirigido - Ciclos Biogeoquímicos.}
\author{Jefferson Campos - RA 423114}
\date{Dec, 20 2011}

\begin{document}

\maketitle
\newpage

\section{Estudo Dirigido - Ciclos Biogeoquímicos.}

\begin{enumerate}
\item A definição ecológica de ciclo biogeoquímico leva em conta o processo de \textcolor{green}{fluxo} de elementos químicos com a participação de \textcolor{green}{organismos} \textcolor{green}{vivos} e etapas \textcolor{green}{de transformação} de escala planetária, além de diversas \textcolor{green}{transformações} químicas.
\item Os compartimentos em que os elementos químicos estão presentes são a \textcolor{green}{atmosfera}, a \textcolor{green}{litosfera} e a \textcolor{green}{hidrosfera}.
\item A \textcolor{green}{hidrosfera} é o compartimento em que os compostos químicos como nitratos e fosfatos encontram-se dissolvidos e pode ser representada pelos \textcolor{green}{rios}, \textcolor{green}{lagos} e oceanos.
\item Na biota, os elementos químicos se combinam formando macromoléculas \textcolor{green}{maiores}, como \textcolor{green}{celuilose} e \textcolor{green}{gordura} (C), \textcolor{green}{proteínas} (N) e \textcolor{green}{adenosina} \textcolor{green}{trifosfato}(P).
\item No diagrama apresentado acima, as setas claras representam o \textcolor{green}{fluxo de energia} e a \textcolor{green}{ciclagem} de \textcolor{green}{nutrientes}. A matéria orgânica é representada por \textcolor{green}{setas escuras} e a inorgânica por \textcolor{green}{setas brancas}. Durante os processos vitais, produtores, consumidores e decompositores \textcolor{green}{perdem} \textcolor{green}{energia} através do calor.
\item Os intercâmbios entre os ambientes terrestres e aquáticos se configuram como importantes mecanismos de balanço na ciclagem de nutrientes. Assim sendo, os processos de \textcolor{green}{absorvição gasosa/nitorgênio}, \textcolor{green}{intemperismo químico de rochas e solo} e \textcolor{green}{absorvição de água} se configuram como entradas no ambiente terrestre e os processos de \textcolor{green}{emissão gasosa}, \textcolor{green}{desnitrificação e outras reações químicas} e \textcolor{green}{perca de água para o subterrâneo} se configuram como saídas do ambiente terrestre. A maior saída do ambiente terrestre e a maior entrada no ambiente aquático é o \textcolor{green}{fluxo superficial} de \textcolor{green}{água}.
\item Um bom exemplo de eficiência na ciclagem de nutrientes são as micorrizas, em que os fungos podem mobilizar \textcolor{green}{fósforo}, \textcolor{green}{potásiso}, \textcolor{green}{cálcio} e \textcolor{green}{magnésio} a partir dos substratos sólidos através da \textcolor{green}{secreção} de \textcolor{green}{ácidos} orgânicos, ficando esses nutrientes disponíveis à planta via \textcolor{green}{micélio} fúngico.
\item O carbono (C) se encontra na atmosfera formando três tipos de compostos, \textcolor{green}{dióxido} de \textcolor{green}{carbono} (CO2), \textcolor{green}{monóxido} de \textcolor{green}{carbono} \ce{(CO)} e \textcolor{green}{metano} \ce{(CH4)}. Na litosfera está presente fazendo parte de rochas na forma de carbonatos metálicos, como o \textcolor{green}{carbonato} de \textcolor{green}{cálcio} \ce{(CaCO3)}, os silicatos de cálcio ou na forma de \textcolor{green}{combustíveis} fósseis. Na hidrosfera aparece \textcolor{green}{carbonato} de \textcolor{green}{cálcio} dissolvido e como íons \textcolor{green}{tricarbonato} \ce{(CO3)} e \textcolor{green}{bicarbonato} \ce{(HCO3)}.
\item Através do processo da \textcolor{green}{respiração}, o carbono, na forma de \ce{CO2}, é retido pelas plantas passando a formar suas moléculas biológicas vitais. Através da \textcolor{green}{transpiração}, o carbono é devolvido para a atmosfera.
\item Atividades humanas como queima de combustíveis fósseis, incêndios florestais e processos industriais são reconhecidas como fontes de \textcolor{green}{dióxido} de \textcolor{green}{carbono}; cabe lembrar que este gás, juntamente com o \textcolor{green}{metano} são dois dos principais gases do efeito \textcolor{green}{estufa}.
\item Apesar de sua abundância entre os gases da atmosfera (78\%), o nitrogênio gasoso ou molecular não pode ser metabolizado pela maioria dos organismos vivos, tornando-se biologicamente ativo após um processo conhecido como \textcolor{green}{fixação} biológica do nitrogênio, que o converte em \textcolor{green}{íon} \textcolor{green}{amônico} \ce{(NH4+)} e \textcolor{green}{nitrato} \ce{(NO3-)}.
\item No meio aquático, as \textcolor{green}{cianobactéria} se encarregam de realizar a fixação biológica do nitrogênio, enquanto que no meio terrestre são as bactérias \textcolor{green}{químiossintética} do gênero \textcolor{green}{Rhizobium}, além das dos gêneros \textcolor{green}{Rhodospirillum} e \textcolor{green}{Pseudomonas} que vivem formando nódulos nas raízes de plantas, especialmente leguminosas, que realizam essa função.
\item No solo, o íon amônio é oxidado por intervenção de bactérias \textcolor{green}{decompositoras} a outras formas químicas como o nitrito e o nitrato, num processo denominado \textcolor{green}{nitrificação}.
\item A \textcolor{green}{desnitrificação} é outro processo importante do ciclo do nitrogênio pelo qual nitratos voltam a ser convertidos em nitrogênio, fechando o ciclo, através da ação de bactérias que catalisam a oxidação da matéria orgânica. A redução do nitrato pode ser completa, originando \textcolor{green}{nitrogênio} gasoso, ou parcial, originando óxidos \textcolor{green}{nitroso}.
\item Há ainda outro importante processo do ciclo do nitrogênio, mediado por organismos decompositores, que atuam sobre excretas de animais e resíduos vegetais transformando os compostos nitrogenados da matéria orgânica em íon amônio \ce{(NH4+)}. Este processo é denominado \textcolor{green}{decomposição}.
\item Os óxidos de nitrogênio gerados a partir da queima de \textcolor{green}{óxidos} \textcolor{green}{nítricos} contaminam o ar e, junto com os óxidos de enxofre, são os componentes básicos das chamadas \textcolor{green}{chuvas} ácidas, que afetam os solos e as águas doces de muitas partes do mundo.
\item Com base no esquema anterior, referente ao ciclo do enxofre, é possível observar que o enxofre pode ser encontrado na forma gasosa através dos compostos \textcolor{green}{dióxido} de \textcolor{green}{enxofre} e \textcolor{green}{ácido} \textcolor{green}{sulfúrico}, tanto na atmosfera ou dissolvido na água.
\item Quando o \textcolor{green}{ácido} \textcolor{green}{sulfúrico}, formado a partir da reação do dióxido de enxofre com a água da atmosfera, chega ao solo, é convertido novamente em sulfatos. Os sulfatos do solo estão, em sua grande maioria, na forma de anidrita (ou sulfato de cálcio, \ce{CaSO4}) e gesso (sulfato de cálcio hidratado, \ce{CaSO4 . 2H2O}), constituindo-se em depósitos marinhos. Dessa forma, o \textcolor{green}{solo/sedimentos - depósito terciário -} é considerado o grande depósito deste elemento.
\item A pirita (FeS2) ou “ouro dos tolos” é formada a partir da combinação do \textcolor{green}{dissulfeto} com o \textcolor{green}{ferro}, nos fundos oceânicos anóxicos. Nos dias de hoje, processos de mineração fazem com que a pirita seja exposta ao meio ambiente e oxidada por bactérias chamadas \textcolor{green}{Thiobacillus}, produzindo ácido sulfúrico.
\item Na natureza, o fósforo encontra-se em sua maior abundância na \textcolor{green}{crosta} \textcolor{green}{terrestre} e nos depósitos de \textcolor{green}{rochas} \textcolor{green}{marinhas}.
\item Nos seres vivos, apesar de em pequenas quantidades, o fósforo desempenha um papel vital, sendo encontrado nos \textcolor{green}{ácidos} \textcolor{green}{nucléicos}, nos \textcolor{green}{fosfolipídeos}, que constituem as membranas de todas as células, além do \textcolor{green}{carboidrato} (\textcolor{green}{hidratos} \textcolor{green}{de carbono}), molécula com ligações altamente energéticas. Além disso, fazem parte de estruturas rígidas como carapaças, ossos e dentes de animais.
\item O ciclo do fósforo difere de outros ciclos como do carbono, do nitrogênio e do enxofre por não possuir formas voláteis que lhe permitem passar dos oceanos à atmosfera e desta para terra firme. Desconsiderando processos de elevação geológica do mar, que podem demandar milhões de anos, a única forma do fósforo chegar ao continente é através das \textcolor{green}{aves} \textcolor{green}{marinhas}, que assimilam-no através das cadeias alimentares marinhas e excretam-no em terra firme, o que se constitui no chamado \textcolor{green}{fixação}.
\item As atividades humanas que mais afetam o ciclo do fósforo são a \textcolor{green}{agricultura}, o abuso de \textcolor{green}{adubagem} e o uso de \textcolor{green}{fertilizantes} com fosfatos.
\item O acúmulo de fósforo em ambientes aquáticos como águas continentais e mares costeiros pode desencadear um processo de aumento de produção com consequente decomposição e desoxigenação das águas profundas, conhecido como \textcolor{green}{eutrofização}.
\item Os íons fosfato no solo são absorvidos pelas \textcolor{green}{plantas}, transformado em macromoléculas e chegam até os animais via \textcolor{green}{cadeia} \textcolor{green}{trófica}. Após morte, ambos sofrem ação dos \textcolor{green}{decompositores} que devolvem o fósforo ao solo e consequentemente à água.
\end{enumerate}
\end{document}
