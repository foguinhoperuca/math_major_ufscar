\documentclass[a4paper,12pt]{article}
\usepackage[brazilian]{babel}
% \usepackage[portuguese]{babel}
% \usepackage[latin1]{inputec}
\usepackage[utf8]{inputenc}
% \usepackage{amssymb}
\usepackage[T1]{fontenc}


\begin{document}

\section{Cibernética.}

\begin{itemize}
\item Definição: comunicação e o controle de máquinas, seres vivos e grupos sociais;
\item Distinção entre robótica e cibernética (confusão com termo CYBORG)
\item Estudo da informação:
\begin{itemize}
\item codificação/descodificação;
\item retroação ou realimentação (feedback);
\item aprendizagem (IA - inteligência artificial);
\end{itemize}
\item Não há distinção entre serves vivos e máquinas;
\item Rompimento com causalidade linear - introdução do mecanismo de feedback
\begin{itemize}
\item Regulação: A --> B, B --> A
\item Base para sistemas autonomos e aprendizado
\end{itemize}
\item Aplicações:
\begin{itemize}
\item militar
\begin{itemize}
\item armas biológicas
\end{itemize}
\item economica
\begin{itemize}
\item US, URSS, França...
\end{itemize}
\end{itemize}
\end{itemize}






% \section{Teoria de Sistemas.}

% \subsection{Histórico.}
% ** proposta em 1937 pelo biólogo Ludwig von Bertalanffy;
% ** A teoria de sistemas, cujos primeiros enunciados datam de 1925, foi  tendo alcançado o seu auge de divulgação na década de 50; 
% ** Em 1956 Ross Ashby introduziu o conceito na ciência cibernética;
% **  A pesquisa de Von Bertalanffy foi baseada numa visão diferente do reducionismo científico até então aplicada pela ciência convencional.

% \subsection{Conceitos.}
% ** Entropia - todo sistema sofre deteriorização;
% ** Sintropia, negentropia ou entropia negativa - para que o sistema continue existindo, tem que desenvolver forças contrárias à Entropia;
% ** Homeostase - capacidade do sistema manter o equilibrio;
% ** Homeorrese - toda vez que há uma ação imprópria (desgaste) do sistema, ele tende a se equilibrar.
% ** Definição de escopo - Sistemas abertos/fechados;
% ** Realimentações (cibernética);

% ** O Sistema é um conjunto de partes interagentes e interdependentes que, conjuntamente, formam um todo unitário com determinado objetivo e efetuam determinada função;
% ** Sistema pode ser definido como um conjunto de elementos interdependentes que interagem com objetivos comuns formando um todo, e onde cada um dos elementos componentes comporta-se, por sua vez, como um sistema cujo resultado é maior do que o resultado que as unidades poderiam ter se funcionassem independentemente. Qualquer conjunto de partes unidas entre si pode ser considerado um sistema, desde que as relações entre as partes e o comportamento do todo sejam o foco de atenção;
% ** Sistema é um conjunto de partes coordenadas, formando um todo complexo ou unitário;

% ** Interdisciplinaridade
% *** Em biologia temos nas células um exemplo, pois não importa quão profundo o estudo individual de um neurônio do cérebro humano, este jamais indicará o estado de uma estrutura de pensamento, se for estirpado, ou morrer, também não alterará o funcionamento do cérebro.
% *** Uma área emergente da biologia molecular moderna que se utiliza bastante dos conceitos da Teoria de Sistemas é a Biologia Sistêmica.
% *** Os mesmos conceitos e princípios que orientam uma organização no ponto de vista sistêmico, estão em todas as disciplinas, físicas, biológicas, tecnológicas, sociológicas, etc. provendo uma base para a sua unificação.

% \section{Modelos Computacionais.}

% * Linguagem R
% ** http://www.r-project.org/
% ** Foi criada originalmente por Ross Ihaka e por Robert Gentleman no departamento de Estatística da universidade de Auckland, Nova Zelândia, e foi desenvolvido por um esforço colaborativo de pessoas em vários locais do mundo.

% \section{Ferramentas.}
% * Fluxograma (exemplo - sites em anexo)
% * UML (exemplo .png em anexo)
% * Modelos Matemáticos
% ** Função (função exponencial - crescimento de uma bactéria - exemplo sites em anexo)

% Bibliografia complementar:

% * [2006] MUNARI, Antônio César de Barros - Notas de Aula - Disciplina: Teoria de Sistemas (TS). Disciplina ministrada no curso de PROCESSAMENTO DE DADOS a Faculdade de Tecnologia de Sorocaba (FATEC-SO).

% * http://www.unicamp.br/fea/ortega/eco/simul/indice-p.htm

% * http://pt.wikipedia.org/wiki/Cibern%C3%A9tica
%   ** GEROVITCH, Slava. "Cybernetics and Information Theory in the United States, France and the Soviet Union". In: WALKER, Mark (Dir.). Science and Ideology: a comparative history. Londres: Rotledge, 2003. Disponível em: <http://www.infoamerica.org/documentos_word/shannon-wiener.htm>.
%   ** MEDINA, Eden. "Designing Freedom, Regulating a Nation: Socialist Cybernetics in Allende’s Chile". Journal of Latin American Studies, n. 38, 2006. Disponível em: <http://informatics.indiana.edu/edenm/EdenMedinaJLASAugust2006.pdf>.
%   ** WIENER, Norbert. Cibernética e sociedade: o uso humano de seres humanos. São Paulo: Cultrix, 1968.

% * http://pt.wikipedia.org/wiki/Teoria_de_sistemas
%   ** Facets of Systems Science; KLIR, George J.; Springer Verlag; 1994
%   ** Introdução à Teoria Geral da Administração; CHIAVENATO, Idalberto; Ed. Makron Books
%   ** Teoria Geral dos Sistemas; BERTALANFFY, Ludwig Von.; Ed. Vozes;1975.
%   ** A Teia da Vida; CAPRA, Fritjof; Ed. Cultrix; 1997.

% * http://www.r-project.org/

\end{document}