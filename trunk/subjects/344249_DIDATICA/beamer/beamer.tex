%%%%%%%%%%%%%%%%%%%%%%%%%%%%%%%%%%%%%%%%%%%%%%%%%%%%%%%%%%%%%%%%%%%%%%%%%%%%%
% 26/05/2010
% edited by Bill Lampos
%
% Feel free to use (copy) the structure (latex formatting source code)
% but not the content of this document.
%
%%%%%%%%%%%%%%%%%%%%%%%%%%%%%%%%%%%%%%%%%%%%%%%%%%%%%%%%%%%%%%%%%%%%%%%%%%%%%
\documentclass[compress,red]{beamer}
\mode<presentation>

\usetheme{Pittsburgh}
% other themes: AnnArbor, Antibes, Bergen, Berkeley, Berlin, Boadilla, boxes, CambridgeUS, Copenhagen, Darmstadt, default, Dresden, Frankfurt, Goettingen,
% Hannover, Ilmenau, JuanLesPins, Luebeck, Madrid, Maloe, Marburg, Montpellier, PaloAlto, Pittsburg, Rochester, Singapore, Szeged, classic

%\usecolortheme{wolverine}
% color themes: albatross, beaver, beetle, crane, default, dolphin, dov, fly, lily, orchid, rose, seagull, seahorse, sidebartab, structure, whale, wolverine

%\usefonttheme{structurebold}
% font themes: default, professionalfonts, serif, structurebold, structureitalicserif, structuresmallcapsserif

% pdf is displayed in full screen mode automatically
%\hypersetup{pdfpagemode=FullScreen}

% define your own colours:
\definecolor{Red}{rgb}{1,0,0}
\definecolor{Blue}{rgb}{0,0,1}
\definecolor{Green}{rgb}{0,1,0}
\definecolor{magenta}{rgb}{1,0,.6}
\definecolor{lightblue}{rgb}{0,.5,1}
\definecolor{lightpurple}{rgb}{.6,.4,1}
\definecolor{gold}{rgb}{.6,.5,0}
\definecolor{orange}{rgb}{1,0.4,0}
\definecolor{hotpink}{rgb}{1,0,0.5}
\definecolor{newcolor2}{rgb}{.5,.3,.5}
\definecolor{newcolor}{rgb}{0,.3,1}
\definecolor{newcolor3}{rgb}{1,0,.35}
\definecolor{darkgreen1}{rgb}{0, .35, 0}
\definecolor{darkgreen}{rgb}{0, .6, 0}
\definecolor{darkred}{rgb}{.75,0,0}

\xdefinecolor{olive}{cmyk}{0.64,0,0.95,0.4}
\xdefinecolor{purpleish}{cmyk}{0.75,0.75,0,0}

% \usepackage{beamerinnertheme_______}
% inner themes include circles, default, inmargin, rectangles, rounded

%\usepackage{beamerouterthemesmoothbars}
% outer themes include default, infolines, miniframes, shadow, sidebar, smoothbars, smoothtree, split, tree

\useoutertheme[subsection=false]{smoothbars}

% to have the same footer on all slides
%\setbeamertemplate{footline}[text line]{xxx xxx xxx}
%\setbeamertemplate{footline}[text line]{} % or empty footer

% include packages
\usepackage[utf8]{inputenc}
\usepackage[brazil]{babel}
\usepackage{amsmath}
\usepackage{epsfig}
\usepackage{graphicx}
\usepackage{url}
\usepackage{multimedia}
\usepackage{hyperref}
\usepackage{setspace}

\title{\Huge{Análise do Plano de Aula sobre Planejamento Financeiro e Juros.}}
\author{Jefferson Campos \\
}
%\institute{{\tiny orientado por}\\ \vspace{.10cm}Ana Paula}
\date{\scriptsize UFSCar -- Universidade Federal de São Carlos(Sorocaba)\\ \vspace{.10cm}}

\begin{document}

\frame{
	\titlepage
}

\section[Índice]{}
\frame{\tableofcontents}

\section{Tema Escolhido.}
\frame{\frametitle{Tema Escolido.}
\indent{\indent{\textbf{ O plano de aula escolhido foi o do planejamento financeiro e cálculo de juros.}}}
}

\section{Indícios de Abordagem Pedagógica Assumida.}
\frame{\frametitle{Indícios de Abordagem Pedagógica Assumida.}
\begin{enumerate}
 \item Subordinação da teoria à prática; \pause
 \item Introdução da teoria {$\to$} discussão e contribuiçoes {$\to$} foco na prática; \pause
 \item Participação ativa do aluno {$\to$} gerar o seu próprio planejamento; \pause
\end{enumerate}
}

\section{Relação Professor e Aluno.}
\frame{\frametitle{Relação Professor e Aluno.}
\begin{enumerate}
 \item Professor como mediador; \pause
 \item Aluno Participa ativamente da discussão; \pause
\end{enumerate}
}

\section{Objetivos.}
\frame{\frametitle{Objetivos.}
\begin{enumerate}
 \item Reflexão sobre o dinheiro e seu uso; \pause
 \item Mostrar a importância do planejamento financeiro pessoal; \pause
 \item Efetuar cálculos com foco na importância de economizar, gastar apenas aquilo que é necessário. - Empregar nos cálculos juros simples e composto; \pause
 \item Construir diversos tipos de planilhas de gastos (utilizando planilhas eletrônicas); \pause
\end{enumerate}
}

\section{Estratégias de Ensino.}
\frame{\frametitle{Estratégia de Ensino.}
\textbf{Prática apoiada pela teoria}
\begin{enumerate}
\item Exposição dos principais tópicos do tema aos alunos, tendo em mente a sua aplicação; \pause
\item Dicussão sobre os tópicos apresentados. Contribuição dos alunos com suas próprias experiências; \pause
\item Prática da teoria exposta anteriormente através da implementação de um planejamento financeiro pessoal para cada aluno baseado em sua realidade. O refinamento será dado através das discussões em sala de aula; \pause
\item Uso, preferencialmente, de uma planilha eletrônica evidenciando a vantagem de uso da mesma. \pause
\end{enumerate}
}
 
\section{Avaliação.}
\frame{\frametitle{Avaliação.}
\textbf{Avaliação dos alunos deve ser feita com base na participação do mesmo. Especificamente:}
\begin{enumerate}
 \item Os alunos são capazes de identificar gastos recorrentes? \pause
 \item Os alunos são capazes de identificar qual o melhor investimento, levando em consideração o prazo, renda e juros? \pause
 \item Os alunos são capazes de determinar o que pode ser investido e o que pode ser poupado? \pause
\end{enumerate}
}

\section{Possibilidades de Desenvolvimento da Proposta.}
\frame{\frametitle{Possibilidades.}
\begin{enumerate}
 \item Enfatizar a importância do controle financeiro; \pause
 \item Discussão sobre juros, prazos e implicações de um endividamento elevado; \pause
 \item Destaque para a prática: colocar as despesas no papel (ou em uma planilha eletrônica) e trabalhar os resultados obtifos com os alunos; \pause
\end{enumerate}
}

\frame{\frametitle{}
\centering{\Huge{\textbf{ Perguntas?}} \vspace{1.10cm}}\\ 
%\centering{Jefferson Campos} \\
\tiny{\textbf{foguinho.peruca@gmail.com}} \\
\centering{\tiny{\textbf{foguinho.peruca@gmail.com}}} \\
}



\bibliographystyle{plain}
\bibliography{ref}
\end{document} 