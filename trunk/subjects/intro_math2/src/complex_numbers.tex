\documentclass[a4paper,12pt]{article}
\usepackage[brazil]{babel}
%% \usepackage[latin1]{inputec}
\usepackage{amssymb}

\begin{document}

\section{Conjunto dos Números Complexos.}

O conjunto dos números complexos foi inventado em 1500, aproximadamente, com o intuito de suprir a necessidade de complementar o modelo de resolução de equações do 3 grau.
A partir de dois pares ordenados $$(a, b)$$ e $$(c, d)$$ do produto cartesiano $$\mathbb{R} \times \mathbb{R} = {(x, y) | x \belong \mathbb{R} e y \in \mathbb{R}}$$ e onde as seguintes condições são válidas:

\begin{enumerate}
\item Igualdade de pares ordenados: $$(a, b) = (c, d) \Leftrightarrow a = c e b = d$$
\item Adição de pares ordenados: $$(a, b) + (c, d) = (a + c, b + d)$$
\item Multiplicação de pares ordenados: $$(a, b) \cdot (c, d) = (ac - bd, ad + bc)$$
\end{enumerate}

É definido o conjunto dos números complexos, denotado por $$\mathbb{C} = $$.

\section{Forma Algébrica.}

Para $$m, n$$ sendo números reais $$\in \\mathbb(R)$$ temos:
\begin{enumerate}
\item $$(m, 0) = (n, 0) \Leftrightarrow m = n$$
\item $$(m, 0) + (n, 0) = (m + n, 0 + 0) \Lefarrow (m + n, 0)$$
\item $$(m, 0) \cdot (n, 0) = (m \cdot n - 0 \cdot 0, m \cdot 0 + n \cdot 0) \Lefarrow (m \cdot n, 0)$$
\end{enumerate}

É possível notar que o segundo elemento é zero nos três casos e o primeiro elemento é sempre um número real. Logo, a seguinte afirmação é verdadeira:
$$(m, 0) = m$$

Sabendo que a equação $$x^2 + 1 = 0$$ \textbf{não} possui raiz real. Entretanto, a raiz para equação deve ser um número multiplicado por ele mesmo que resulte em $$-1$$. Para tanto é possível calcular, levando em consideração o conjunto dos números complexos, exposto logo acima:
$$$$

$$z = a + bi$$
$$a$$ (parte real) e $$b$$ (parte imaginaria) e $$i$$ = $$\sqrt{-1}$$

Igualdade: $$a + bi = c +di$$ $$\Leftrightarrow$$ $$a = c$$ e $$b = d$$

Imaginario puro quando $$a = 0$$ e $$b = 0$$.

Real quando $$b = 0$$.

\subsection{Classificação Dos Números Complexos.}



\section{Potências de i.}

% See if ''I'' can be replaced by some latex's command.
\section{Operações I - Adição, Subtração e Multiplicação.}

\subsection{Operacoes.}

\section{Conjugado.}

\section{Operações II - Divisão.}

\section{Representação Geométrica.}

\subsection{Módulo.}

\subsection{Argumento.}

\section{Forma Trigonométrica.}

\section{Potenciação.}


%% Classe's notes - professor Adilson Brandão - UFSCAR/SOROCABA
\section{Interpretação Geométrica.}

\section{Propriedades}
\begin{description}
\item[Comutativa]
\item[Associativa]
\item[Elemento Neutro]
\item[Elemento Oposto]
\end{description}

\section{Conjugado.}

\section{Common Problrem.}

\subsection{Exemplo Numérico.}

\section{Divisão}

\subsection{\mathbb{R}}}
\subsection{\mathbb{C}}}

\section{Proposição I.}
\begin{itemize}
\item[-]
\item[-]
\end{itemize}

\section{Corolário.}

\section{Exercício.}


\newcounter{quest}
\begin{list}{\textbf{Questão \arabic{quest}.}}{\usecounter{quest}
\setlength{\labelwidth}{-2mm} \setlength{\parsep}{0mm}
\setlength{\topsep}{0mm} \setlength{\leftmargin}{0mm}}
\renewcommand{\labelenumi}{(\alph{enumi})}
\item Existe algum padrão para $$i^0, i^1, i^2, i^3, i^4, i^5, i^6 \ldots ?$$
\begin{enumerate}
\item Caso exista, descreva-o generalizando.
\item 
\end{enumerate}
\item Calcule $$1 + i + i^2 + i^3 + \ldots + i^{2011}}}$$.
 ́
a
\end{list}



%% Example - useless
$$\delta = -\frac{1}{4}$$ (menor valor para x)

\end{document}

